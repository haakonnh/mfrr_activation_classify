\newpage
\section{Conclusion}
This study has examined the feasibility and relevance of short-term forecasting
of manual Frequency Restoration Reserve (mFRR) energy activation direction at a
15-minute resolution following the Nordic power system’s transition to automated
mFRR activation on 4 March 2025. Motivated by the structural change in balancing
market operations and the resulting increase in temporal granularity, the
analysis adopts a participant-centric perspective, focusing on the information
constraints faced by market participants at gate closure.

Using data from the Norwegian NO1 bidding zone, a supervised multi-class
classification framework was developed to predict mFRR activation
direction—up-regulation, down-regulation, or no activation—based solely on
participant-feasible information. Extensive feature engineering was performed
across market prices, cross-zonal flows, production and load forecasts, and
temporal activation indicators. Tree-based machine-learning models were trained
and evaluated using temporally consistent training, validation, and test splits,
and benchmarked against a persistence-based naïve baseline to ensure realistic
performance assessment.

The results demonstrate that short-term mFRR activation direction is predictable
to a meaningful extent at a 15-minute horizon under realistic participant
information constraints. A key finding is that the majority of predictive power
arises from short-term temporal persistence in activation patterns, indicating
that recent activation states provide strong signals about near-term balancing
behavior. While additional system and market features contribute incremental
information, their influence is secondary relative to persistence-based
indicators.

The findings further indicate that explicit forecasting of activation direction
is both practically and methodologically valuable. From a practical standpoint,
correctly anticipating activation direction constitutes a critical short-term
signal for market participants—particularly demand-side aggregators of flexible
resources—where correct directional predictions can yield meaningful economic
value independently of precise volume or price forecasts. From a methodological
perspective, activation direction forecasting serves as a useful intermediate
step toward conditional forecasting of activation volumes and prices, mitigating
challenges associated with zero-inflation and mixed activation regimes.

By providing one of the first empirical studies of participant-feasible mFRR
activation direction forecasting at 15-minute resolution under the new Nordic
market design, this work contributes evidence on short-term predictability in
automated balancing markets. The results establish a foundation for future work
on conditional, probabilistic, and decision-integrated forecasting frameworks
aimed at supporting data-driven participation in evolving balancing markets.