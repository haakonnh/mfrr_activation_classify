\newpage
\section{Conclusion}
This project investigated short-horizon prediction of mFRR activations in NO1
under participant-relevant timing constraints.

The results show that activation dynamics are strongly auto-correlated, making a
persistence-based baseline highly competitive. More complex models can add
incremental value, but improvements are constrained by class imbalance, and
limited information available at bid close.

Overall, the findings suggest that modelling activation direction explicitly is
a practical framing for market participants, and that future improvements are
most likely to come from better handling of rare up-activation events and from
integrating higher-quality near-real-time signals.
