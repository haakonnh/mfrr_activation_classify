\newpage
\section*{Appendix}\label{sec:appendix}

\subsection*{AI Declaration}
In this project, AI tools were utilized in three primary ways. 
Firstly, AI-assisted coding tools, more specifcially GitHub Copilot, were used 
to aid in the programming process. These tools provided suggestions through 
autocomplete features, which were then reviewed and modified. All code 
generated by AI tools was thoroughly evaluated and adjusted to ensure correctness.


Secondly, AI language models, specifically ChatGPT, were employed to help
refine the language and structure of certain sections of the text. In an increasingly 
expanding report, AI tools were used to evaluate and suggest improvements to the
structure and clarity of the writing. Spelling and grammar checks were also performed
using AI tools to enhance the overall quality of the text.


Finally, AI tools were used to assist in the search of relevant literature. 
Google Scholar Labs, an AI-powered literature search tool, was used to explore 
relevant sources for the literature review and other sections of the report.
This tool was simply used for the search process, as subsequent evaluation of
the relevance and quality of the sources was made without AI assistance. 


An important consideration when using AI tools in project work is the potential
of over-reliance on these tools, which can lead to a lack of critical thinking
and also possible derailment from the original intent of the work. To mitigate
this, it is important to avoid viewing AI tools as factual sources, and
instead treat them as suggestive tools that require oversight. This was the 
approach taken in this project. 