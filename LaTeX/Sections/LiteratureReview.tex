\newpage
\section{Literature Review}\label{sec:literature-review} This chapter reviews
literature relevant to balancing market forcasting, first outlining the unique
characteristics of balancing activation markets and the challenges they present.
Then existing methodologies  
for handling these challenges are presented, before finally identifying specific
research gaps that this study and future work can address.


\subsection{mFRR Energy Activation Market Characteristics}
Balancing markets are by nature unpredictable, as their primary function is to
maintain system stability in the face of unforeseen imbalances between supply
and demand. This inherent unpredictability is in most balancing markets handled
through the capacity market mechanism. The mFRR energy activation market
distinguishes itself by only compensating participants for actual energy
delivered during activation events. Participants must also bid in the correct
direction of activation (upward or downward) to be eligible for activation. When
an up-regulating activation is required, the up-regulation price is by design
higher than the day-ahead market price, and vice versa for down-regulating
activations \autocite{klaeboeDayaheadMarketBidding2022}. These market
characteristics make it lucrative for participants to predict activation events
accurately, as successful predictions can lead to significant financial gains. 


In 2022, Klæboe et al. \autocite{klaeboeDayaheadMarketBidding2022} analyzed
day-ahead market bidding strategies for flexible generators taking the balancing
power market into account. They found near-zero gains from incorporating
balancing market predictions into day-ahead bidding strategies. They discuss,
however, that the need for balancing services will increase in the future, and
that such strategies will therefore become more relevant and profitable. In
Svenska Kraftnät's balancing market outlook 2030
\autocite{BalancingMarketOutlook} they present that the mFRR capacity demand has
and will steadily increase. The report also suggests that since the automated
mFRR EAM is only an intermediate step for connecting to MARI (Manually Activated
Reserves Initiative), which is an upcoming European-wide mFRR market, further
increases in mFRR demand are to be expected. This suggests that predicting mFRR
activations will become increasingly important for market participants seeking
to optimize their market strategies.


The mFRR energy activation market transitioned from an hourly to a 15-minute
resolution as of 4th March 2025, an endeavor aimed at enhancing market
efficiency and integrating renewable energy sources more effectively
\autocite{Transition15minuteMarket}. Under the previous hourly structure,
activation signals were constrained to coarse discrete time blocks. Thus,
short-lived imbalances or, for instance, rapid ramps in renewable generation
could not be reflected optimally in activation decisions. moving to a 15-minute
resolution reduces this discretization effect
\autocite{kallsetImprovingBalancingActivation2025}.


A study by Kallset and Farahmand found that increased resolution significantly
reduces such structural imbalances and achieves about 60\% of the possible
reduction in total balancing, compared to a 5-minute resolution ideal
\autocite{kallsetImprovingBalancingActivation2025}. Their findings imply that
imbalances are now corrected more accurately and efficiently on shorteer time
scales, making activation patterns more sensitive to rapid system changes.
Consequently, the dynamics of the mFRR energy activation market have become more
granular and potentially more volatile, increasing the relevance, but also the
difficulty, of short-term activation forecasting. 

\subsection{Activation Uncertainty in Balancing Market Forecasting}

Most forecasting studies in the balancing-market literature focus on predicting
continuous system variables such as imbalance volumes or imbalance prices. These
quantities are natural targets for system operators, who must minimize balancing
costs and anticipate system stress. However, imbalance volumes are inherently
conditional on the discrete activation direction—upward, downward, or
none—because volume magnitudes reflect both the sign and size of the underlying
imbalance. When activation direction is not modelled explicitly, directional
uncertainty becomes embedded in the volume forecast itself, contributing to
higher forecast errors precisely around directional switches. This section
reviews the main approaches to imbalance-volume forecasting, highlighting their
strengths and limitations before discussing how the literature has attempted to
treat activation-direction uncertainty explicitly or implicitly.

Most forecasting studies in the balancing-market literature focus on predicting
continuous system variables such as imbalance volumes or imbalance prices. These
quantities are natural targets for system operators, who must minimize balancing
costs and anticipate system stress. However, imbalance volumes are inherently
conditional on the discrete activation direction—upward, downward, or
none—because volume magnitudes reflect both the sign and size of the underlying
imbalance. When activation direction is not modelled explicitly, directional
uncertainty becomes embedded in the volume forecast itself, contributing to
noisy forecasts. 

To address this challenge, the literature proposes various modelling approaches
for representing activation uncertainty. These approaches differ in how
uncertainty is represented. For clarity, these methods are grouped into six
families: (i) scenario-based activation models, which simulate possible future
imbalance trajectories; (ii) activation-ratio or expected-activation models,
which derive expected activation ratios from historical data; (iii)
activation-probability and chance-constraint models, which enforce reliability
requirements based on probabilistic imbalance representations; (iv)
activation-range or interval-uncertainty models, which define bounded sets of
feasible activation magnitudes; (v) regressor-based activation models, including
machine-learning methods that explicitly predict activation direction; and (vi)
Markov activation models, which represent activation direction as a stochastic
process represented by transition probabilities.

The following sections review these modelling families, beginning with
imbalance-volume forecasting studies before assessing how each
uncertainty-modelling approach handles, or fails to handle, the discrete
up/down/none activation decision relevant for mFRR energy markets.

\subsubsection{Direct Imbalance Volume Forecasting}

At the system-operator end of the spectrum, a substantial body of literature
focuses on point forecasting of continuous imbalance volumes, with the primary
purpose of improving TSO operational decisions. Singh et al.
\autocite{singhNovelDynamicImbalance2025} exemplify this class of work through a
regression-based model for short-term imbalance forecasting in Belgium. They
argue that increasing renewable variability, combined with the 15-minute
activation window, necessitates accurate short-horizon forecasts to allow TSOs
to anticipate system deviations more effectively. Their best-performing model
reduces balancing costs by 44.51\% relative to TSO benchmarks, driven by
reductions in energy-not-supplied, excess energy, and correction costs.

Related work in the Nordic region highlights similar challenges. Edling and
Azarang (2025) forecast short-term mFRR activation volumes across the four
Swedish bidding zones using LSTM models
\autocite{azarangMachineLearningBasedPrediction}. Their results reveal strong
geographical heterogeneity in predictability: SE2 exhibits comparatively high
accuracy, while SE3 and SE4 show limited predictability due to the prevalence of
zero-activation intervals. This distinction highlights an important structural
feature of the Nordic system: regions with frequent imbalances do not
necessarily experience frequent activations. Because Sweden's flexible
hydropower capacity is concentrated in SE1 and SE2, the TSO often activates
reserves there even when imbalances originate in other zones, provided network
constraints permit it. Earlier results by Overmaat
\autocite{overmaatBalancingContributionsNordic} confirm that SE1 and SE2
historically provide the majority of balancing energy on short and medium time
scales.

While Edling and Azarang approach the problem from a TSO perspective, Backe et
al. in the KoBas project \autocite{backePredictionsPricesVolumes2023} examine
imbalance-volume forecasting from a market-participant-oriented standpoint. They
also develop a LSTM model, including several Nordic bidding zones. Their
analysis yields three relevant insights. First, balancing volumes are relatively
auto-correlated: past imbalances contain predictive information about short-term
imbalances. Second, they note that forecast accuracy could likely be improved by
incorporating weather-related variables. Third, they stress that zero-regulation
dominates the dataset, meaning the model must infer relatively infrequent
activation events from a mostly inactive baseline. This is an inherent
limitation when direction is not modelled explicitly.

Plakas et al. \autocite{plakasPredictionImbalancePrices2025} also focus on
market-participant perspectives, proposing a two-stage probabilistic framework
for forecasting imbalance volumes and prices sequentially in the Greek balancing
market. The first stage employs quantile regression to generate probabilistic
forecasts of system imbalances. The second stage leverages the quantiles to
predict imbalance prices. Plakas et al. find that system imbalance volumes are
critical predictors of imbalance prices, underscoring the correlation between
these two variables. By extending from imbalance volume point forecasts to price
forecasting, Plakas et al. provide more actionable insights for market
participants seeking to capitalize on opportunities in the balancing market.
Their results show that imbalance volumes strongly influence imbalance prices,
indicating that system-state indicators provide valuable information for bidders
seeking to anticipate balancing-market outcomes.

In a similar vein, Bankefors (2024) applies linear machine-learning-based time
series models (ARIMAX, SARIMAX) to predict signed mFRR activation volumes
\autocite{bankeforsDayaheadModellingElectricity}. Bankefors concludes that while
imbalance and activation volume forecasting is challenging, the models showed
promise in implicitly predicting activation direction. He suggests that future
work could explore classification-based approaches to directly predict
activation direction rather than inferring it from volume forecasts.

Together, these studies highlight a structural limitation of direct imbalance
volume forecasting: the sign and magnitude of imbalances volumes depend on
activation direction, meaning that models that do not explicitly model
activation direction uncertainty must implicitly learn it from the noisy data.
This can lead to degraded forecast quality, especially around directional
switches.  While Plakas et al. \autocite{plakasPredictionImbalancePrices2025}
couple imbalance volume and price forecasting in a two-stage framework, this
approach could potentially be extended further by first predicting activation
direction, then imbalance volumes, and finally imbalance prices. The subsequent
sections discuss how existing literature has addressed the challenge of modeling
activation direction uncertainty. 


\subsubsection{Scenario-Based Activation Models}
Scenarios are often used to handle uncertainty in optimization problems.
Possible future outucomes are represented as discrete scenarios, each with an
associated probability. In the context of balancing markets, scenarios represent
possible trajectories of net system imbalance, which implicitly determine
required activation volumes. This approach is for example used in reserve
dimensioning \autocite{khodadadiInvestigatingReserveDimensioning} and stochastic
scheduling or bidding frameworks
\autocite{hagstromOptimizingRiskAwareBidding2025}.

In \autocite{habergStochasticMixedInteger2017}, Håberg and Doorman model
activation uncertainty using three discrete scenarios: high, median, and low, as
illustrated in figure \ref{fig:imbalance_scenarios}. The scenarios are
represented as forecasted continuous imbalance volumes over a 40-minute horizon.
The imbalance forecast scenarios were generated from probability distributions
based on historical imbalance data. 

\begin{figure}[H]
    \centering
    \includegraphics[width=0.7\textwidth]{Images/imbalance_scenarios_so.png}
    \caption{Imbalance forecast scenarios
    \autocite{habergStochasticMixedInteger2017}.}
    \label{fig:imbalance_scenarios}
\end{figure}

A key limitation of this modelling family is that activation direction is not
modelled explicitly, but arises solely from the sign of the scenario imbalance
volumes. Consequently, any uncertainty in direction is entirely dependent on the
quality of the scenario-generation process. Thus, they rely on high quality
imbalance forecasting models, which is, as outlined in this literature review, a
challenging task in itself. All in all, scenario-based approaches seem to be
useful only when highly accurate imbalance forecasts are available. Therefore,
they do not directly address the challenge of modelling activation direction
uncertainty. \textbf{Hmm, actually I do not really feel like this section fits
here in the structure. I may have misunderstood what you mean by scenario-based
approach, but it seems it is not really a modeling of activation uncertainty,
activation direction is simply implied by the sign of imbalance scenarios which
arise from imbalance forecasting. }




\subsubsection{Activation Ratio or Expected Activation}
Some studies model activation uncertainty using constant activation ratios or
expected activations. A common approach is to estimate the probability of
activation in each direction (upward, downward, none) based on historical
activation frequencies. Irrmann (2023) applied this method analyze and model the
Nordic balancing markets \autocite{irrmannAnalysisModellingBalancing2023}. In
this study, \textit{regulation states} (up, down, none) are sampled based on
historical frequencies, before activation volumes are drawn from a modelled
distribution conditonal on the sampled state. This approach decouples the
discrete activation decision from the continuous volume forecasting, allowing
for more targeted modelling of each component. However, the activation
probabilities are statically estimated from historical frequencies. Although
such estimations may be adequate over longer time horizons, they are likely to
perform poorly in the short term, as activation patterns are highly
non-stationary and dependent on the current system state, requiring more
adaptive methods. Irrmann somewhat addressed this by estimating separate
probabilities for each month of the year, but this coarse temporal segmentation
is unlikely to capture the full dynamics of activation behaviour. 

\subsubsection{Activation Probability and Chance Constraints}
Chance constraints are a mathematical optimization technique used to handle
uncertainty by ensuring that certain constraints are satisfied with a specified
probability. Papavasiliou et al. (2022)
\autocite{papavasiliouMultiAreaReserveDimensioning2022} apply chance-constrained
optimization to the problem of reserve dimensioning in a multi-area power
system. Here, uncertainty described by scenarios are revealed in the form of
continuous imbalances. The chance constraints impose reliability limits for up
and downward reserves, ensuring that the procured reserves can cover imbalances
with a certain probability. This is an application of chance constraints to
balancing markets, but it is geared towards TSO reserve dimensioning rather than
participant-side activation forecasting. Additionally, activation direction is,
similar to Håberg and Doorman's scenario-based approach
\autocite{habergStochasticMixedInteger2017}, only modelled implicitly through
the sign of the continuous imbalance scenarios. 

Browell (2018) \autocite{RiskConstrainedTrading} develops risk constrained
trading strategies for stochastic generators in the UK balancing market. In this
study, \textit{system length}, i.e. the net imbalance direction, is modelled
probabilistically by a logistic regression model. Then, chance, or risk,
constraints are imposed to ensure that trading strategies meet certain
performance criteria with high probability. This study is tailored particularly
to stochastic generators, whose production uncertainty directly influences their
balancing market participation. Thus, the method and results will not generalize
perfectly to other applications, but this paper marks an early and important
attempt to explicitly model activation/imbalance direction uncertainty.  



\subsubsection{Activation Uncertainty Ranges}
Pavíc et al. (2023) argue that deterministic reserve activation models
inccurately represent the activation uncertainty. Thus, they present a
stochastic model, but more interestingly, they also propose a robust electric
vehicle aggregator scheduling model using uncertain bounded activation ranges
\autocite{pavicElectricVehicleAggregator2023}. They use \textit{reserve
activation} (RA) as input for activation uncertainty, which is defined as the
ratio of activated reserve energy to the accepted reserve capacity. Their
analysis is limited to 30-minute FCR and aFRR reserve data for 2018. Activation
data is gathered and probability distributions are constructed as figure
\ref{fig:activation_uncertainty_ranges}, representing the likelihood of
different activation ratios. Relevant statistics, like mean, max and quantiles,
are used as inputs for their models. 

\begin{figure}[H]
    \centering
    \includegraphics[width=0.7\textwidth]{Images/uncertainty_ranges.png}
    \caption{Activation ratio uncertainty ranges for aFRR up
    \autocite{pavicElectricVehicleAggregator2023}.}
    \label{fig:activation_uncertainty_ranges}
\end{figure}

Using activation ranges to represent uncertainty is an interesting approach, as
it directly models the fraction of accepted reserves that are likely to be
activated. This is very useful information for flexible demand-side aggregators,
who must decide how much capacity to offer based on expected activations. The
ranges imply worst-case, best-case, and expected activation scenarios, which can
be used to inform robust models attempting to remain feasible under uncertainty.
Such uncertainty sets may be more appropriate than probabilistic and
deterministic frameworks for flexible demand-side market participants, who must
ensure feasibility at all costs. Pavic et al. operate in the context of FCR and
aFRR reserves, where there is always an activated imbalance in one direction
\autocite{pavicElectricVehicleAggregator2023}. mFRR balancing, on the other
hand, often requires no activation at all. This is not a problem per se, but it
would skew the activation ratio distributions significantly, as a large
probability mass would be located at zero activation. 



\subsubsection{Markov Activation Models}
Klæboe et al. benchmarked time series based forecasting models for electricity
balancing markets in 2015 \autocite{klaeboeBenchmarkingTimeSeries2015}. In this
study, they separate relevant work into two families: models explicitly modeling
balancing state and those modeling it implicitly. Various implicit and explicit
activation direction models have been discussed extensively in this literature
review, but Klæboe et al. highlight Markov models as a particularly interesting
approach for explicit balance state modeling. The study refers to work by Olsson
and Söder, who used a non-time-homogeneous Markov model, with varying transition
probabilities depending on balancing state durations. Balancing states of
durations 0-5 hours were modeled with separate transition matrices, while longer
durations used a common matrix \autocite{olssonModelingRealTimeBalancing2008}.

Another approach is to construct different transition matrices for different
hours in the day. This allows the model to capture patterns such as higher
transition probabilities during the day compared to night. Klæboe et al. also
tested a model based on the work of Croston
\autocite{crostonForecastingStockControl1972}, only distinguishing between
activation and no activation, thereby discounting direction. This model
distinguishes between up- and down regulations in a sepearate price- or volume
process. 

One-hour ahead predictions were shown to be quite accurate, predicting correctly
63\% amd 73\% of the time for duration dependent and hour specific models,
respectively. However, the models struggled with longer horizons, with accuracy
dropping to around 30\% at day-ahead. The Croston-based model, when benchmarked
on regulation vs no-regulation, achieved around 59\% accuracy at one-hour ahead,
notably lower than the other models, but outperformed them at day-ahead
prediction. Another finding was the struggle to predict direct transitions
between upward and downward activations, as these events have low transition
rates. The duration dependent Markov model is an interesting approach, and one
that is used as inspiration for the model development in this study. It captures
persistence in activation patterns, which are known to be important
\autocite{backePredictionsPricesVolumes2023a}.

\subsubsection{Regressor-Based Activation Models }

Whereas imbalance forecasting estimates continuous system imbalance magnitudes,
activation-direction forecasting seeks to predict the discrete TSO decision to
activate upward, downward, or no mFRR energy. Activation direction is directly
relevant for market participants because bids must be placed in the correct
direction to be eligible for activation. With the Nordic system's transition to
15-minute activation intervals, short-term direction forecasting has become more
important. However, the academic literature that treats direction as a primary
target remains sparse.

Svedlindh and Yngveson \autocite{svedlindhPriceFormationForecasting2025} examine
the general price formation in intraday and mFRR markets. Among other
explorations, they develop logistic regression and ANN (Artificial Neural
Network) models to predict activation direction in the mFRR activation market.
The ANN model outperforms the logistic regression, achieving solid
\textit{accuracy} and \textit{F1-scores}. They identify, however, that
\textit{class imbalance} poses a significant challenge, as no-activation events
dominate the dataset. This imbalance skews model performance, making it
difficult to accurately predict the less frequent upward and downward
activations. Despite these acknowledged challenges, Svedlindh and Yngveson
achieve promising results. They find that mFRR capacity market prices and
procured volumes are informative predictors of activation direction. 

Porras (2025) \autocite{porrasShortTermForecastingMFRR} applies an XGBoost
two-stage model to sequentially forecast activation direction and imbalance
prices at hourly resolution in SE2. The study demonstrates the potential of
tree-based methods, but it also exposes two practical limitations for
participant-oriented forecasting: (i) the model operates at hourly resolution,
and it remains unclear how well the approach would perform at the new 15-minute
resolution; and (ii) its most important predictor is “balance-direction at
\(t-0\)” that appears to be unavailable to market participants at the time of
bidding. This represents a form of feature leakage (use of variables that would
not be accessible in real decision-making) and likely inflates performance.
\textbf{This is correct, right?}

In summary, regressor-based approaches show promise for explicit direction
forecasting, but current studies leave questions yet to be answered: Can
direction be predicted reliably at 15-minute resolution with only available
information? And can sparse up/down events be predicted with useful precision?
The present study addresses these questions by evaluating multiple classifier
families under strict participant-feasible information constraints and at the
updated 15-minute resolution.

\subsubsection{Literature Synthesis}
The reviewed studies differ in their forecasting targets, methodological
approaches, and intended users. Regression-based studies such as Singh et al.
\autocite{singhNovelDynamicImbalance2025} and Edling \& Azarang
\autocite{azarangMachineLearningBasedPrediction} focus on point forecasts of
system imbalance volumes. These models primarily serve system operators and rely
on data streams that are not always available to market participants. While such
point forecasts provide valuable information about the magnitude of expected
imbalances, they do not directly predict activation direction and do not
quantify uncertainty.

Probabilistic approaches, exemplified by Plakas et al.
\autocite{plakasPredictionImbalancePrices2025}, attempt to characterize
uncertainty explicitly by producing distributions over imbalance volumes and
prices. This is more directly useful for market participants, but their
framework till does not address the discrete up/down/none decision that governs
mFRR activations.

Only a small number of studies examine activation direction itself. Svedlindh
and Yngveson \autocite{svedlindhPriceFormationForecasting2025} develop logistic
regression and ANN classifiers, showing that activation direction is in
principle predictable but heavily constrained by class imbalance and the use of
capacity-market features that are not available in real-time to market
participants. Porras \autocite{porrasShortTermForecastingMFRR} investigates
activation direction using XGBoost at an hourly resolution in SE2, demonstrating
the potential of machine-learning classification but leaving open questions
regarding finer temporal resolution and the feasibility of predictions outside
high-activation regions.

Taken together, the literature indicates that imbalance information carries
predictive value for prices and activations, but existing approaches either (i)
focus on continuous volume forecasting, (ii) rely on non-participant data, or
(iii) operate at coarser temporal scales than current Nordic market design.
There remains limited evidence on whether activation direction can be predicted
reliably using only features available to market participants and at the new
15-minute resolution.

\begin{table}[!htbp]
\centering
\caption{Comparison of key studies on balancing market forecasting}
\label{tab:literature-comparison}
\begin{adjustbox}{width=\textwidth}
\begin{tabular}{l l c c}
\toprule
\textbf{Study} & \textbf{Target Variable} & \textbf{Resolution} & \textbf{Model
Type} \\
\midrule
Singh et al.\ (2025) & Imbalance volume & 15-min & Regression \\
Edling \& Azarang (2025) & mFRR activation volume & 1-hour & LSTM \\
Plakas et al.\ (2025) & Imbalance volume + price & 1-hour & Probabilistic / QR
\\
Svedlindh \& Yngveson (2025) & Activation direction & 1-hour & Logistic / ANN \\
Porras (2025) & Activation direction + price & 1-hour & XGBoost \\
\bottomrule
\end{tabular}
\end{adjustbox}
\end{table}
Table \ref{tab:literature-comparison} summarizes the key characteristics of the
reviewed studies, reviewed studies, highlighting their differing targets,
resolutions, and model types.


\subsubsection{Research Gap}
Despite notable progress in balancing-market forecasting, several gaps remain in
the existing literature. Most existing studies focus on continuous imbalance or
activation volume forecasting at an hourly resolution. While these approaches
provide valuable insights for system operators, they do not address the discrete
up/down/none activation decisions faced by market participants, especially under
the newer 15-minute market design.

Studies that do consider activation direction (Svedlindh \& Yngveson and Porras)
do so at an hourly resolution. With the transition to 15-minute activation
intervals, however, finer temporal patterns and short-lived imbalances may
become more relevant. Investigating whether higher time resolution can improve
prediction performance is therefore of interest.

Geographically, relatively little work has focused on the Norwegian bidding
zones. The reviewed studies investigate the Belgian and Greek systems (Singh et
al. and Plakas et al.) or Swedish bidding zones (Edling \& Azarang, Svedlindh \&
Yngveson, and Porras). Swedish bidding zones naturally share similarities with
Norwegian zones, as both are part of the Nordic system. However, differences in
generation mixes, cross-zonal flows, and local balancing tendencies may lead to
distinct activation patterns.

A further consideration is that few studies robustly address the data
availability constraints faced by market participants. Some studies incorporate
TSO-only data, while others do not explicitly evaluate whether their chosen
features would be accessible to market participants at decision time. As a
result, the extent to which activation direction can be predicted under
participant-feasible information constraints remains relatively open.

Taken together, these observations suggest the value of a systematic evaluation
of activation-direction forecasting in the NO1 bidding zone, using features
available to market participants and operating at the updated 15-minute
resolution. This study contributes to this area by comparing a range of
machine-learning models and assessing their performance under realistic data and
operational constraints.