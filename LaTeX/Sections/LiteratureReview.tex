\newpage
\section{Literature Review}\label{sec:literature-review}
This chapter reviews literature relevant to balancing market forcasting, first 
outlining the unique characteristics of 
balancing activation markets and the challenges they present. Then existing methodologies  
for handling these challenges are presented, before finally
identifying specific research gaps that this study and future work can address.


\subsection{LOOKING AT PROPOSED ACTIVATION UNCERTAINTY VARIANTS}

\begin{itemize}
    \item \textbf{Scenario-Based Activation Models}: Model uncertainty as scenarios.
    Stochastic optimization. Håberg \& Doorman (2017) - reserve dimensioning, TSO
    perspective \autocite{habergStochasticMixedInteger2017}.
    \\ Chance-constrained, imbalance scenario simulations, reserve allocation and
    dimensioning \autocite{khodadadiInvestigatingReserveDimensioning}.
    \\ Looks at introducing mFRR capacity market participation for hydropower
    as it was new at the time. Two-stage stochastic participation in day-ahead market,
    mFRR capacity and energy. Uncertainty in prices modelled with scenarios
    \autocite{khodadadiOptimalCoordinatedHydropower2021}.
    \item \textbf{Activation Ratio or Expected Activation}:
    

\end{itemize}

\subsection{mFRR Energy Activation Market Characteristics}
Balancing markets are by nature unpredictable, as their primary function is to
maintain system stability in the face of unforeseen imbalances between supply
and demand. This inherent unpredictability is in most balancing markets handled
through the capacity market mechanism. The mFRR energy activation market distinguishes 
itself by only compensating participants for actual energy delivered during
activation events. Participants must also bid in the correct direction of
activation (upward or downward) to be eligible for activation. When an up-regulating
activation is required, the up-regulation price is by design higher than the
day-ahead market price, and vice versa for down-regulating activations 
\autocite{klaeboeDayaheadMarketBidding2022}. These market
characteristics make it lucrative for participants to predict activation events 
accurately, as successful predictions can lead to significant financial gains. 


In 2022, Klæboe et al. \autocite{klaeboeDayaheadMarketBidding2022} analyzed 
day-ahead market bidding strategies for flexible generators
taking the balancing power market into
account. They found near-zero gains from incorporating balancing market
predictions into day-ahead bidding strategies. They discuss, however, that
the need for balancing services will increase in the future, and that such
strategies will therefore become more relevant and profitable. In Svenska Kraftnät's
balancing market outlook 2030 \autocite{BalancingMarketOutlook} they present
that the mFRR capacity demand has and will steadily increase. The report also suggests
that since the automated mFRR EAM is only an intermediate step for connecting to MARI
(Manually Activated Reserves Initiative), which is an upcoming European-wide mFRR 
market, further increases in mFRR demand are to be expected. This suggests that
predicting mFRR activations will become increasingly important for market participants
seeking to optimize their market strategies.


The mFRR energy activation market transitioned from an hourly to a 15-minute
resolution as of 4th March 2025, an endeavor aimed at enhancing market efficiency
and integrating renewable energy sources more effectively \autocite{Transition15minuteMarket}.
Under the previous hourly structure, activation signals were constrained to coarse
discrete time blocks. Thus, short-lived imbalances or, for instance, rapid ramps in
renewable generation could not be reflected optimally in activation decisions. moving
to a 15-minute resolution reduces this discretization effect 
\autocite{kallsetImprovingBalancingActivation2025}.


A study by Kallset and Farahmand 
found that increased resolution significantly reduces such structural imbalances and
achieves about 60\% of the possible reduction in total balancing, compared to a
5-minute resolution ideal \autocite{kallsetImprovingBalancingActivation2025}. Their
findings imply that imbalances are now corrected more accurately and efficiently on
shorteer time scales, making activation patterns more sensitive to rapid system changes.
Consequently, the dynamics of the mFRR energy activation market have become more 
granular and potentially more volatile, increasing the relevance, but also the difficulty,
of short-term activation forecasting. 


\subsection{Existing Approaches to Balancing Activation Market Forecasting}
Forecasting in balancing markets has developed along two main methodological paths. 
The first focuses on predicting continuous system variables such as imbalance volumes 
or imbalance prices, mostly appealing to system operators. The second 
path targets the discrete activation decisions made in the mFRR energy activation 
market, which are directly relevant for market participants who must place bids in the 
correct direction. The following sections review key contributions in both areas, 
highlighting their methodological approaches, limitations, and relevance for short-term 
activation forecasting under the current 15-minute Nordic market design.

\subsubsection{Imbalance Volume and Price Forecasting}
Imbalance volume forecasting plays a central role in balancing markets 
\autocite{singhNovelDynamicImbalance2025} \autocite{plakasPredictionImbalancePrices2025}. 
At the system
level, transmission system operators (TSOs) rely on accurate imbalance volume forecasts
to minimize 
balancing costs and ensure system stability. As market participants, 
accurate imbalance volume predictions are valuable, as this can inform short-term 
bidding strategies in balancing markets, particularly in markets with high 
renewable penetration. Consequently, a substantial body of 
recent research has focused on developing models that forecast system imbalance over
short-horizons. 

At the system-operator end of the spectrum, several studies develop point forecasting
aimed primarily at optimizing TSO operations. One example is the work of 
Singh et al. \autocite{singhNovelDynamicImbalance2025}, who
propose a regression-based approach for short-term forecasting of imbalance
volumes. They argue that increased renewable variability, combined with the
15-minute activation window, necessitates accurate short-horizon forecasts to
enable transmission system operators to anticipate and manage imbalances more
effectively. One of their key findings is that their best-performing model 
reduces the balancing costs by 44.51\% compared to TSO-based forecasting. The lower
forecast errors lead to reduced costs of energy not supplied, excess energy, and
corrections. 

While Singh et al. obtain important results for short-term imbalance forecasting
in the Belgian power system, related work examines imbalance
forecasting in the Nordic region, where renewable mixes and activation patterns may
differ. Edling \& Azarang (2025) analyze short-term mFRR activation volumes across
the four Swedish bidding zones using LSTM-based (Long Short-Term Memory) regression
models \autocite{azarangMachineLearningBasedPrediction}. Their study highlights strong
geographical hetereogeneity in activation patterns, identifying structural differences 
between bidding zones. Price region SE2 exhibited the best forecasting accuracy, while
zones SE3 and SE4 showed limited predictability due to high zero-activation frequencies.
This finding spotlights an important consideration: regions with frequent imbalance,
for instance due to a high share of wind power, 
do not necessarily have high activation frequencies. SE2 appears to have
 high activation frequencies
due to its significant share of Sweden's flexible hydropower resources, which are often
called upon to balance the system. Thus, if an imbalance occurs in SE4, the TSO may
choose to activate reserves in SE2 if it is more economical and transmission constraints
allow it. Overmaat's study on balancing contributions in the Nordic electricity system,
though relatively dated (2019), underscores this point further by concluding that SE1
and SE2 dominate the Swedish balancing contributions on the short and medium time scales
\autocite{overmaatBalancingContributionsNordic}. 

Both Singh et al. and Edling \& Azarang aim to provide system operators with 
actionable insights and tools to better forecast and manage imbalances.
While better TSO forecasting does not alter physical imbalance maggnitudes, it improves
the timing and precision of activation decisions, making activation volumes more
closely reflect the real-time system state. For market participants, this suggests
that accurate system imbalance forecasts may offer predictive power regarding 
coming activations. This perspective is developed further in participant-oriented
studies such as Plakas et al. 
\autocite{plakasPredictionImbalancePrices2025}, who propose a two-stage probabilistic 
framework for forecasting imbalance
volumes and prices sequentially in the Greek balancing market. The first stage 
employs quantile regression to generate probabilistic forecasts of system imbalances.
The second stage leverages the quantiles to predict imbalance prices. 
Plakas et al. find that system imbalance volumes are critical predictors of imbalance
prices, underscoring the correlation between these two variables. By extending 
from imbalance volume point forecasts to price forecasting, 
Plakas et al. provide more actionable insights for market participants
seeking to capitalize on opportunities in the balancing market. Their results show that 
imbalance volumes strongly influence imbalance prices, indicating that system-state 
indicators provide valuable information for bidders seeking to anticipate balancing-market 
outcomes.

\subsubsection{Activation Forecasting}
Whereas imbalance forecasting estimates system imbalance magnitudes, 
activation forecasting predicts TSO's decision to activate upward, 
downward, or no mFRR energy. Activation direction is directly relevant for market 
participants, as balancing market bids must be placed in the correct direction to be
eligible for activation. As the Nordic system transitions to a 15-minute market 
resolution, short-term activation forecasting has become increasingly important, 
yet the academic literature on this topic remains sparse.

Svedlindh and Yngveson \autocite{svedlindhPriceFormationForecasting2025}
examine the general price formation in intraday and mFRR markets. Among other
explorations, they develop logistic regression and ANN (Artificial Neural Network)
models to predict activation direction in the mFRR activation market. The ANN model
outperforms the logistic regression, achieving solid \textit{accuracy} and 
\textit{F1-scores}. They identify, however, that \textit{class imbalance} 
poses a significant challenge, as no-activation events dominate the dataset. This
imbalance skews model performance, making it difficult to accurately predict the less 
frequent upward and downward activations. Despite these acknowledged challenges, Svedlindh
and Yngveson achieve promising results. They find that mFRR capacity market prices 
and procured volumes are informative predictors of activation direction. Since the 
capacity market is cleared a day ahead with hourly procurement, its outcomes appear to 
correlate with subsequent activation behaviour.

A recent study in 2025, Porras
\autocite{porrasShortTermForecastingMFRR} investigated the use of
XGBoost for short-term forecasting of mFRR activation direction and imbalance
prices with a two-stage multi-horizon model. This paper is closely
related to the present study, as it also focuses on predicting mFRR activations.
However, Porras leans heavily into XGBoost, while this study explores a broader
range of models. Additionally, Porras produces predictions at an hourly resolution,
whereas this study aims for a finer 15-minute resolution - aligning with the recent
Nordic mFRR market reform. Porras targets the SE2 price area in Sweden, thereby
incorporating data from different sections of the Nordic power system. The
dynamics may be similar, but this study focuses exclusively on the NO1 price area
in Norway, which may exhibit different characteristics. The primal gap in Porras'
study, which this study aims to address, is the opportunity opened by the recent
15-minute market reform, which provides more granular data and potentially
improved prediction capabilities.

\subsubsection{Literature Synthesis}
The reviewed studies differ in their forecasting targets, 
methodological approaches, and intended users. Regression-based studies such 
as Singh et al. \autocite{singhNovelDynamicImbalance2025} and Edling \& Azarang 
\autocite{azarangMachineLearningBasedPrediction} focus on point forecasts of 
system imbalance volumes. These models primarily serve system operators and rely 
on data streams that are not always available to market participants. While such 
point forecasts provide valuable information about the magnitude of expected 
imbalances, they do not directly predict activation direction and do not quantify 
uncertainty.

Probabilistic approaches, exemplified by Plakas et al. 
\autocite{plakasPredictionImbalancePrices2025}, attempt to characterize 
uncertainty explicitly by producing distributions over imbalance volumes and 
prices. This is more directly useful for market participants, but their framework 
till does not address the discrete up/down/none decision that governs mFRR 
activations.

Only a small number of studies examine activation direction itself. Svedlindh and 
Yngveson \autocite{svedlindhPriceFormationForecasting2025} develop logistic 
regression and ANN classifiers, showing that activation direction is in principle 
predictable but heavily constrained by class imbalance and the use of 
capacity-market features that are not available in real-time to market participants.
Porras \autocite{porrasShortTermForecastingMFRR} investigates activation 
direction using XGBoost at an hourly resolution in SE2, demonstrating the potential 
of machine-learning classification but leaving open questions regarding finer 
temporal resolution and the feasibility of predictions outside high-activation 
regions.

Taken together, the literature indicates that imbalance information carries 
predictive value for prices and activations, but existing approaches either 
(i) focus on continuous volume forecasting, (ii) rely on non-participant data, 
or (iii) operate at coarser temporal scales than current Nordic market design. 
There remains limited evidence on whether activation direction can be predicted 
reliably using only features available to market participants and at the new 
15-minute resolution.

\begin{table}[!htbp]
\centering
\caption{Comparison of key studies on balancing market forecasting}
\label{tab:literature-comparison}
\begin{adjustbox}{width=\textwidth}
\begin{tabular}{l l c c}
\toprule
\textbf{Study} & \textbf{Target Variable} & \textbf{Resolution} & \textbf{Model Type} \\
\midrule
Singh et al.\ (2025) & Imbalance volume & 15-min & Regression \\
Edling \& Azarang (2025) & mFRR activation volume & 1-hour & LSTM \\
Plakas et al.\ (2025) & Imbalance volume + price & 1-hour & Probabilistic / QR \\
Svedlindh \& Yngveson (2025) & Activation direction & 1-hour & Logistic / ANN \\
Porras (2025) & Activation direction + price & 1-hour & XGBoost \\
\bottomrule
\end{tabular}
\end{adjustbox}
\end{table}
Table \ref{tab:literature-comparison} summarizes the key characteristics of the reviewed studies,
reviewed studies, highlighting their differing targets, resolutions, and model types.


\subsubsection{Research Gap}
Despite notable progress in balancing-market forecasting, several gaps remain in the 
existing literature.
Most existing studies focus on continuous imbalance or activation volume forecasting 
at an hourly resolution. While these approaches provide valuable insights for system 
operators, they do not address the discrete up/down/none activation decisions faced 
by market participants, especially under the newer 15-minute market design.

Studies that do consider activation direction (Svedlindh \& 
Yngveson and Porras) do so at an hourly resolution. With the transition to 15-minute
activation intervals, however, finer temporal patterns and short-lived imbalances 
may become more relevant. Investigating whether higher time resolution can improve 
prediction performance is therefore of interest.

Geographically, relatively little work has focused on the Norwegian bidding zones. 
The reviewed studies investigate the Belgian and Greek systems (Singh et al. and 
Plakas et al.) or Swedish bidding zones (Edling \& Azarang, Svedlindh \& Yngveson,
and Porras). Swedish bidding zones naturally share similarities with Norwegian zones,
as both are part of the Nordic system. However, differences in generation mixes,
cross-zonal flows, and local balancing tendencies
 may lead to distinct activation patterns.

A further consideration is that few studies robustly address the data availability
constraints faced by market participants. Some studies incorporate TSO-only data, 
while others do not explicitly evaluate whether their chosen features would be accessible
to market participants at decision time.
As a result, the extent to which activation direction can be predicted 
under participant-feasible information constraints remains relatively open.

Taken together, these observations suggest the value of a systematic evaluation of 
activation-direction forecasting in the NO1 bidding zone, using features available to
market participants and operating at the updated 15-minute resolution. This study 
contributes to this area by comparing a range of machine-learning models and assessing 
their performance under realistic data and operational constraints.