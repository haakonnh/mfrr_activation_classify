\newpage
\section{Literature Review}\label{sec:literature-review}
This chapter reviews literature relevant to the prediction of mFRR activations 
in a top-down approach. The review begins by outlining the broader context of
balancing activation markets and the challenges they present. It then delves into 
which methodologies exist for handling these challenges, before finally
identifying specific research gaps that this study aims to address.

\subsection{mFRR Energy Activation Market Characteristics}
Balancing markets are by nature unpredictable as their primary function is to
maintain system stability in the face of unforeseen imbalances between supply
and demand. This inherent unpredictability is handled in most balancing markets
by the capacity market mechanism. The mFRR energy activation market distinguishes 
itself by only compensating participants for actual energy delivered during
activation events. Participants must also bid in the correct direction of
activation (upward or downward) to be eligible for activation. When an up-regulating
activation is required, the up-regulation price is by design higher than the
day-ahead market price, and vice versa for down-regulating activations 
\autocite{klaeboeDayaheadMarketBidding2022}. These market
characteristics make it lucrative to predict activation events accurately, as
successful predictions can lead to significant financial gains. 

In 2022, Klæboe et al. \autocite{klaeboeDayaheadMarketBidding2022} analyzed 
day-ahead market bidding strategies for flexible generators
taking the balancing power market into
account. They found near-zero gains from incorporating balancing market
predictions into day-ahead bidding strategies. They discuss, however, that
the need for balancing services will increase in the future, and that such
strategies will therefore become more profitable. In Svenska Kraftnät's
balancing market outlook 2030 \autocite{BalancingMarketOutlook} they present
that the mFRR capacity demand has and will steadily increase. \textit{Source}

The mFRR energy activation market transitioned from an hourly to a 15-minute
resolution as of 4th March 2025, an endeavor aimed at enhancing market efficiency
and integrating renewable energy sources more effectively \autocite{Transition15minuteMarket}.
An hourly resolution was deemed overly discrete and can miss intra-hour dynamics; a 
15-minute resolution reduces discretization and better approximates the continuous 
variation of supply, demand, and balancing needs. A study by Kallset and Farahmand 
found that increased resolution significantly reduces such structural imbalances and
achieves about 60\% of the possible reduction in total balancing, compared to a
5-minute resolution ideal \autocite{kallsetImprovingBalancingActivation2025}. Thus 
higher resolution make balancing activations align more closely with actual real-
time system needs, and is less restricted by the coarse time blocks of hourly markets.











\subsection{Existing Approaches to Balancing Activation Market Forecasting}
This chapter reviews relevant literature on balancing market forecasting. 


\subsubsection{Imbalance Volume and Price Forecasting}
Imbalance volume forecasting plays a central role in balancing markets 
\autocite{singhNovelDynamicImbalance2025} \autocite{plakasPredictionImbalancePrices2025}. 
At the system
level, transmission system operators (TSOs) rely on accurate imbalance volume forecasts
to minimize 
balancing costs and ensure system stability. As market participants, 
accurate imbalance volume predictions are valuable, as this can inform short-term 
bidding strategies in balancing markets, particularly in markets with high 
renewable penetration. Consequently, a substantial body of 
recent research has focused on developing models that forecast system imbalance over
short-horizons. 

At the system-operator end of the spectrum, several studies develop point forecasting
aimed primarily at optimizing TSO operations. One example is the work of 
Singh et al. \autocite{singhNovelDynamicImbalance2025}, who
propose a regression-based approach for short-term forecasting of imbalance
volumes. They argue that increased renewable variability, combined with the
15-minute activation window, necessitates accurate short-horizon forecasts to
enable transmission system operators to anticipate and manage imbalances more
effectively. One of their key findings is that their best-performing model 
reduces the balancing costs by 44.51\% compared to TSO-based forecasting. The lower
forecast errors lead to reduced costs of energy not supplied, excess energy, and
corrections. 

While Singh et al. obtain important results for short-term imbalance forecasting
in the Belgian power system, related work examines imbalance
forecasting in the Nordic region, where renewable mixes and activation patterns may
differ. Edling \& Azarang (2025) analyze short-term mFRR activation volumes across
the four Swedish bidding zones using LSTM-based (Long Short-Term Memory) regression
models \autocite{azarangMachineLearningBasedPrediction}. Their study highlights strong
geographical hetereogeneity in activation patterns, identifying structural differences 
between bidding zones. Price region SE2 exhibited the best forecasting accuracy, while
zones SE3 and SE4 showed limited predictability due to high zero-activation frequencies.
This finding spotlights an important consideration: regions with frequent imbalance,
for instance due to a high share of wind power, 
do not necessarily have high activation frequencies. SE2 appears to have
 high activation frequencies
due to its significant share of Sweden's flexible hydropower resources, which are often
called upon to balance the system. Thus, if an imbalance occurs in SE4, the TSO may
choose to activate reserves in SE2 if it is more economical and transmission constraints
allow it. Overmaat's study on balancing contributions in the Nordic electricity system,
though relatively dated (2019), underscores this point further by concluding that SE1
and SE2 dominate the Swedish balancing contributions on the short and medium time scales
\autocite{overmaatBalancingContributionsNordic}. 

Both Singh et al. and Edling \& Azarang aim to provide system operators with 
actionable insights and tools to better forecast and manage imbalances.
While better TSO forecasting does not alter physical imbalance maggnitudes, it improves
the timing and precision of activation decisions, making activation volumes more
closely reflect the real-time system state. For market participants, this suggests
that accurate system imbalance forecasts may offer predictive power regarding 
coming activations. This perspective is developed further in participant-oriented
studies such as Plakas et al. 
\autocite{plakasPredictionImbalancePrices2025}, who propose a two-stage probabilistic 
framework for forecasting imbalance
volumes and prices sequentially in the Greek balancing market. The first stage 
employs quantile regression to generate probabilistic forecasts of system imbalances.
The second stage leverages the quantiles to predict imbalance prices. 
Plakas et al. find that system imbalance volumes are critical predictors of imbalance
prices, underscoring the correlation between these two variables. By extending 
from imbalance volume point forecasts to price forecasting, 
Plakas et al. provide more actionable insights for market participants
seeking to capitalize on opportunities in the balancing market. Their results show that 
imbalance volumes strongly influence imbalance prices, indicating that system-state 
indicators provide valuable information for bidders seeking to anticipate balancing-market 
outcomes.

\subsubsection{Activation Forecasting}
Whereas imbalance forecasting aims to estimate the magnitude of system imbalances, 
activation forecasting focuses on predicting the TSO's decision to activate upward, 
downward, or no mFRR energy. Activation direction is directly relevant for market 
participants, as balancing market bids must be placed in the correct direction to be
eligible for activation. As the Nordic system transitions to a 15-minute market 
resolution, short-term activation forecasting has become increasingly important, 
yet the academic literature on this topic remains sparse.

Svedlindh and Yngveson \autocite{svedlindhPriceFormationForecasting2025}
explore the general price formation in intraday and mFRR markets. Among other
explorations, they develop logistic regression and ANN (Artificial Neural Network)
models to predict activation direction in the mFRR activation market. The ANN model
outperforms the logistic regression, achieving solid \textit{accuracy} and 
\textit{F1-scores}. They identify, however, that \textit{class imbalance} 
poses a significant challenge, as no-activation events dominate the dataset. This
imbalance skews model performance, making it difficult to accurately predict the less 
frequent upward and downward activations. Despite these acknowledged challenges, Svedlindh
and Yngveson achieve above expected performance, seemingly sustained by the inclusion
of mFRR capacity market cleared volumes and prices. Cleared mFRR capacity market data
is not necessarily
available to market participants in real-time, and thus limits the practical applicability
of their models. \textbf{Is this true Jay? As far as I know, we do not have access
to capacity market data (volumes, prices). They do not explain how the data 
is gathered.}

A recent study in 2025, Porras
\autocite{porrasShortTermForecastingMFRR} investigated the use of
XGBoost for short-term forecasting of mFRR activation direction and imbalance
prices with a two-stage multi-horizon model. This paper is closely
related to the present study, as it also focuses on predicting mFRR activations.
However, Porras leans heavily into XGBoost, while this study explores a broader
range of models. Additionally, Porras produces predictions at an hourly resolution,
whereas this study aims for a finer 15-minute resolution - aligning with the recent
Nordic mFRR market reform. Porras targets the SE2 price area in Sweden, thereby
incorporating data from different sections of the Nordic power system. The
dynamics may be similar, but this study focuses exclusively on the NO1 price area
in Norway, which may exhibit different characteristics. The primal gap in Porras'
study, which this study aims to address, is the opportunity opened by the recent
15-minute market reform, which provides more granular data and potentially
improved prediction capabilities.

\subsubsection{Methodological Distinctions Across Studies}
In its most basic form, regression can be used to make \textit{point forecasts}
of future values - providing a single expected value for each time step. Regression
can also be extended to produce \textit{probabilistic distributions}, thus capturing
the uncertainty inherent in balancing markets.



\subsection{Research Gap}
While existing literature has made significant strides in predicting balancing
market activations, several gaps remain. Notably, there is a lack of studies
that leverage and discuss potential impacts of the recent transition to 15-minute 
time intervals in the Nordic
mFRR markets. This reform presents an opportunity to enhance prediction accuracy
by utilizing more granular data. Additionally, while various machine learning
techniques have been explored, there is a need for a comprehensive comparison
of different models specifically tailored to the unique characteristics of the
mFRR energy activation market. Many of the recent studies on this topic have
focused on regions with high activation volumes and frequencies, such as SE2 in 
Sweden. However, there is a lack of research focusing on regions like NO1 in Norway,
where activations are less frequent. While SE2 and other high-activation regions 
most likely would yield better prediction results due to more balanced classes,
it is still valuable to explore the challenges and opportunities in regions with
lower activation frequencies. This study aims to fill these gaps by focusing on
the NO1 price area in Norway, utilizing the new 15-minute market data, and
comparing a range of machine learning models to identify the most effective
approaches for predicting mFRR activations.
