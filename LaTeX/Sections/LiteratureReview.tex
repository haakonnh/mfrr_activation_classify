\newpage
\section{Literature Review}\label{sec:literature-review}
This chapter reviews the literature relevant to balancing-market forecasting. It
first outlines the unique characteristics of balancing activation markets and
the challenges they present. Existing methodologies for handling these
challenges are then presented, before finally identifying specific research gaps
that this study and future work can address.


\subsection{mFRR Energy Activation Market Characteristics}
The mFRR energy activation market distinguishes itself from the capacity market
by only compensating participants for actual energy delivered during activation
events. It is thus a pay-as-produced market, where participants are compensated
based on the volume of energy they deliver when activated by the TSO
\autocite{esettNordicImbalanceSettlement2024}. When an up-regulating activation
is required, the up-regulation price is by design higher than the day-ahead
market price, and vice versa for down-regulating activations
\autocite{klaeboeDayaheadMarketBidding2022}. These market characteristics make
it lucrative for participants to predict activation events accurately, as
successful predictions can lead to significant financial gains. 


In 2022, Klæboe et al. \autocite{klaeboeDayaheadMarketBidding2022} analyzed
day-ahead market bidding strategies for flexible generators taking the balancing
power market into account. They found near-zero gains from incorporating
balancing market predictions into day-ahead bidding strategies. They note,
however, that the need for balancing services will increase in the future, and
that such strategies will therefore become more relevant and profitable. In
Svenska Kraftnät's balancing market outlook 2030
\autocite{BalancingMarketOutlook} they report that mFRR capacity demand has
increased and will continue to increase. The report also suggests that since the automated
mFRR EAM is only an intermediate step for connecting to MARI (Manually Activated
Reserves Initiative), which is an upcoming European-wide mFRR market, further
increases in mFRR demand are to be expected. This suggests that predicting mFRR
activations will become increasingly important for market participants seeking
to optimize their market strategies.


The mFRR energy activation market transitioned from an hourly to a 15-minute
resolution as of 4 March 2025, an endeavour aimed at enhancing market
efficiency and integrating renewable energy sources more effectively
\autocite{Transition15minuteMarket}. Under the previous hourly structure,
activation signals were constrained to coarse discrete time blocks. Thus,
short-lived imbalances or, for instance, rapid ramps in renewable generation
could not be reflected optimally in activation decisions. Moving to a 15-minute
resolution reduces this discretization effect
\autocite{kallsetImprovingBalancingActivation2025}.


A study by Kallset and Farahmand found that increased resolution significantly
reduces such structural imbalances and achieves about 60\% of the possible
reduction in total balancing, compared to a 5-minute resolution ideal
\autocite{kallsetImprovingBalancingActivation2025}. Their findings imply that
imbalances are now corrected more accurately and efficiently on shorter time
scales, making activation patterns more sensitive to rapid system changes.
Consequently, the dynamics of the mFRR energy activation market have become more
granular and potentially more volatile, increasing the relevance, but also the
difficulty, of short-term activation forecasting. 

\subsection{Balancing Market Forecasting}
For mFRR EAM market participants, the most relevant target is often the
activated energy volumes. This target is crucial, as it directly informs bidding
strategies and operational decisions. However, activated volumes are conditional
on direction because bids must be placed in the actually activated direction to
be eligible for activation. In practice, the data are also strongly
zero-inflated, meaning most intervals have no activation, and the distribution
of non-zero volumes is different in up- and down-regulating events. When
direction is not modelled explicitly, the target distribution becomes a mixture
of up-, down-, and no-activation regimes, which can inflate apparent noise and
reduce the usefulness of forecasts for decision-making. Consequently, modelling
activation-direction uncertainty explicitly is an important component of
balancing-market forecasting.


To address this challenge, the literature proposes various modelling approaches
for representing activation uncertainty. These approaches differ in how
uncertainty is represented. For clarity, these methods are grouped into six
families:
\begin{enumerate}
    \item Scenario-based activation models, which simulate possible future imbalance trajectories;
    \item Activation-ratio or expected-activation models, which derive expected activation ratios from historical data;
    \item Activation-probability and chance-constraint models, which enforce reliability requirements based on probabilistic imbalance representations;
    \item Activation-range or interval-uncertainty models, which define bounded sets of feasible activation magnitudes;
    \item Regressor-based activation models, including machine-learning methods that explicitly predict activation direction; and
    \item Markov activation models, which represent activation direction as a stochastic process characterized by transition probabilities.
\end{enumerate}

The following sections review these modelling families, beginning with direct
imbalance-volume forecasting studies, before assessing how each
uncertainty-modelling approach handles, or fails to handle, the discrete
up/down/none activation decision relevant for mFRR energy markets.

\subsubsection{Direct Imbalance and Activation Volume Forecasting}

\textbf{Differentiate imbalance-activation}

At the system-operator end of the spectrum, a substantial body of literature
focuses on point forecasting of continuous system \textit{imbalance} volumes,
with the primary purpose of improving TSO operational decisions. Singh et al.
\autocite{singhNovelDynamicImbalance2025} exemplify this class of work through a
regression-based model for short-term imbalance forecasting in Belgium. They
argue that increasing renewable variability, combined with the 15-minute
activation window, necessitates accurate short-horizon forecasts to allow TSOs
to anticipate system deviations more effectively. Their best-performing model
reduces balancing costs by 44.51\% relative to TSO benchmarks, driven by
reductions in energy-not-supplied, excess energy, and correction costs.

Related work in the Nordic region highlights similar challenges. Edling and
Azarang (2025) forecast short-term mFRR \textit{activation} volumes across the four
Swedish bidding zones using LSTM models
\autocite{azarangMachineLearningBasedPrediction}. Their results reveal strong
geographical heterogeneity in predictability: SE2 exhibits comparatively high
accuracy, while SE3 and SE4 show limited predictability due to the prevalence of
zero-activation intervals. This distinction highlights an important structural
feature of the Nordic system: regions with frequent imbalances do not
necessarily experience frequent activations. Because Sweden's flexible
hydropower capacity is concentrated in SE1 and SE2, the TSO often activates
reserves there even when imbalances originate in other zones, provided network
constraints permit it. Earlier results by Overmaat
\autocite{overmaatBalancingContributionsNordic} confirm that SE1 and SE2
historically provide the majority of balancing energy on short and medium time
scales.

While Edling and Azarang approach the problem from a TSO perspective, Backe et
al. in the KoBas project \autocite{backePredictionsPricesVolumes2023} examine
activation volume forecasting from a market-participant-oriented standpoint.
They also develop a LSTM model covering several Nordic bidding zones. Their
analysis yields three relevant insights. First, balancing volumes are relatively
autocorrelated: past activations contain predictive information about short-term
activations. Second, they note that forecast accuracy could likely be improved
by incorporating weather-related variables. Third, they stress that
zero-regulation dominates the dataset, meaning the model must infer relatively
infrequent activation events from a mostly inactive baseline. \textbf{This is an
inherent limitation when direction is not modelled explicitly.} 

Plakas et al. \autocite{plakasPredictionImbalancePrices2025} focus on
market-participant perspectives as well, proposing a two-stage probabilistic
framework for forecasting imbalance volumes and prices sequentially in the Greek
balancing market. The first stage employs quantile regression to generate
probabilistic forecasts of system imbalances. The second stage leverages the
quantiles to predict imbalance prices. Plakas et al. find that system imbalance
volumes are critical predictors of imbalance prices, underscoring the
correlation between these two variables. By extending from imbalance volume
point forecasts to price forecasting, Plakas et al. provide more actionable
insights for market participants seeking to capitalize on opportunities in the
balancing market. Their results show that imbalance volumes strongly influence
imbalance prices, indicating that system-state indicators provide valuable
information for bidders seeking to anticipate balancing-market outcomes.

In a similar vein, Bankefors (2024) applies linear machine-learning-based time
series models (ARIMAX, SARIMAX) to predict signed mFRR activation volumes
\autocite{bankeforsDayaheadModellingElectricity}. Bankefors concludes that while
activation volume forecasting is challenging, the models showed
promise in implicitly predicting activation direction. He suggests that future
work could explore classification-based approaches to directly predict
activation direction rather than inferring it from volume forecasts.

Together, these studies highlight a structural limitation of direct imbalance
volume forecasting: the sign and magnitude of imbalances depend on activation
direction, meaning that models that do not explicitly model activation direction
uncertainty must implicitly learn it from noisy, zero-inflated data. This can
degrade forecast quality, especially around directional switches, and it can
encourage regression models to predict values close to zero most of the time.
These limitations motivate work that treats activation uncertainty as
a modelling concern. The subsequent sections review how existing literature has
addressed activation uncertainty.


\subsubsection{Scenario-Based Models}
Scenarios are often used to handle uncertainty in optimization problems.
Possible future outcomes are represented as discrete scenarios, each with an
associated probability. In the context of balancing markets, scenarios represent
possible trajectories of net system imbalance, which implicitly determine
required activation volumes. This approach is, for example, used in reserve
dimensioning \autocite{khodadadiInvestigatingReserveDimensioning} and stochastic
scheduling or bidding frameworks
\autocite{hagstromOptimizingRiskAwareBidding2025}.

In \autocite{habergStochasticMixedInteger2017}, Håberg and Doorman model
activation uncertainty using three discrete scenarios: high, median, and low, as
illustrated in Figure \ref{fig:imbalance_scenarios}. The scenarios are
represented as forecasted continuous imbalance volumes over a 40-minute horizon.
The imbalance forecast scenarios were generated from probability distributions
based on historical imbalance data. 

\begin{figure}[H]
    \centering
    \includegraphics[width=0.7\textwidth]{Images/imbalance_scenarios_so.png}
    \caption{Imbalance forecast scenarios.}
    \label{fig:imbalance_scenarios}
\end{figure}

Figure \ref{fig:imbalance_scenarios} is reproduced from H\aa{}berg and Doorman
\autocite{habergStochasticMixedInteger2017}.

\textbf{A key limitation of this modelling family is that activation direction is not
modelled explicitly, but arises solely from the sign of the scenario imbalance
volumes.} Consequently, any uncertainty in direction is entirely dependent on the
quality of the scenario-generation process. Thus, they rely on high-quality
imbalance forecasting models, which, as outlined in this literature review, are
challenging to develop. Overall, scenario-based approaches seem to be useful
only when highly accurate imbalance forecasts are available. Therefore,
they do not directly address the challenge of modelling activation direction
uncertainty. 


\subsubsection{Activation Uncertainty Modelling Approaches}



\paragraph{Activation Ratio or Expected Activation.}
Some studies model activation uncertainty using constant activation ratios or
expected activations. A common approach is to estimate the probability of
activation in each direction (upward, downward, none) based on historical
activation frequencies. Irrmann (2023) applied this method to analyze and model the
Nordic balancing markets \autocite{irrmannAnalysisModellingBalancing2023}. In
this study, \textit{regulation states} (up, down, none) are sampled based on
historical frequencies, before activation volumes are drawn from a modelled
distribution conditional on the sampled state. This approach decouples the
discrete activation decision from the continuous volume forecasting, allowing
for more targeted modelling of each component. However, the activation
probabilities are statically estimated from historical frequencies. Although
such estimations may be adequate over longer time horizons, they are unlikely to
be representative of short-term activation behaviour. Irrmann somewhat addressed
this by estimating separate probabilities for each month of the year, but this
coarse temporal segmentation is unlikely to capture the full dynamics.

\paragraph{Activation Probability and Chance Constraints.}
Chance constraints are a mathematical optimization technique used to
handle uncertainty by ensuring that certain constraints are satisfied with a
specified probability. Papavasiliou et al. (2022)
\autocite{papavasiliouMultiAreaReserveDimensioning2022} apply chance-constrained
optimization to the problem of reserve dimensioning in a multi-area power
system. Here, uncertainty described by scenarios is revealed in the form of
continuous imbalances. The chance constraints impose reliability limits for up
and downward reserves, ensuring that the procured reserves can cover imbalances
with a certain probability. This is an application of chance constraints to
balancing markets, but it is geared towards TSO reserve dimensioning rather than
participant-side activation forecasting. Additionally, activation direction is,
similar to Håberg and Doorman's scenario-based approach
\autocite{habergStochasticMixedInteger2017}, only modelled implicitly through
the sign of the continuous imbalance scenarios. 

Browell (2018) \autocite{RiskConstrainedTrading} develops risk-constrained
trading strategies for stochastic generators in the UK balancing market. In this
study, \textit{system length}, i.e. the net imbalance direction, is modelled
probabilistically by a logistic regression model. Then, chance, or risk,
constraints are imposed to ensure that trading strategies meet certain
performance criteria with high probability. This study is tailored particularly
to stochastic generators, whose production uncertainty directly influences their
balancing market participation. Thus, the method and results will not generalize
perfectly to other applications, but this paper marks an early and important
attempt to explicitly model activation/imbalance direction uncertainty.  



\paragraph{Activation Uncertainty Ranges.}
Pavi\'c et al. (2023) argue that deterministic reserve activation models
inaccurately represent activation uncertainty. Thus, they present a
stochastic model, but more interestingly, they also propose a robust electric
vehicle aggregator scheduling model using uncertain bounded activation ranges
\autocite{pavicElectricVehicleAggregator2023}. They use \textit{reserve
activation} (RA) as input for activation uncertainty, which is defined as the
ratio of activated reserve energy to the accepted reserve capacity. Their
analysis is limited to 30-minute FCR and aFRR reserve data for 2018. Activation
data are gathered and probability distributions are constructed as in Figure
\ref{fig:activation_uncertainty_ranges}, representing the likelihood of
different activation ratios. Relevant statistics, such as the mean, max, and
quantiles, are used as inputs for their models.

\begin{figure}[H]
    \centering
    \includegraphics[width=0.7\textwidth]{Images/uncertainty_ranges.png}
    \caption{Activation ratio uncertainty ranges for aFRR up.}
    \label{fig:activation_uncertainty_ranges}
\end{figure}

Figure \ref{fig:activation_uncertainty_ranges} is reproduced from Pav\'ic et al.
\autocite{pavicElectricVehicleAggregator2023}.

Using activation ranges to represent uncertainty is an interesting approach, as
it directly models the fraction of accepted reserves that are likely to be
activated. This is very useful information for flexible demand-side aggregators,
who must decide how much capacity to offer based on expected activations. The
ranges imply worst-case, best-case, and expected activation scenarios, which can
be used to inform robust models attempting to remain feasible under uncertainty.
Such uncertainty sets may be more appropriate than probabilistic and
deterministic frameworks for flexible demand-side market participants, who must
ensure feasibility at all costs. Pavi\'c et al. operate in the context of FCR and
aFRR reserves, where there is always an activated imbalance in one direction
\autocite{pavicElectricVehicleAggregator2023}. mFRR balancing, on the other
hand, often requires no activation at all. This is not a problem per se, but it
would skew the activation ratio distributions significantly, as a large
probability mass would be located at zero activation. 



\paragraph{Markov Activation Models.}
In 2015 time-series-based forecasting models for electricity balancing markets
were benchmarked \autocite{klaeboeBenchmarkingTimeSeries2015}. In this study,
relevant work is separated into two families: models explicitly modelling
balancing state and those modelling it implicitly. Various implicit and explicit
activation direction models have been discussed extensively in this literature
review, but here Markov models are highlighted as a particularly interesting
approach for explicit balance-state modelling. The study refers to a
non-time-homogeneous Markov model, with varying transition probabilities
depending on balancing state durations. Balancing states of durations 0-5 hours
were modelled with separate transition matrices, while longer durations used a
common matrix \autocite{olssonModelingRealTimeBalancing2008}.

Another approach is to construct different transition matrices for different
hours in the day. This allows the model to capture patterns such as higher
transition probabilities during the day compared to night. A model
distinguishing only between activation and no activation, thereby discounting
direction, was also tested, \autocite{crostonForecastingStockControl1972}. This
model distinguishes between up- and down regulations in a separate price- or
volume process. 

One-hour-ahead predictions were shown to be quite accurate, predicting correctly
63\% and 73\% of the time for duration-dependent and hour-specific models,
respectively. However, the models struggled with longer horizons, with accuracy
dropping to around 30\% at day-ahead. The second model, when benchmarked
on regulation vs no-regulation, achieved around 59\% accuracy at one-hour ahead,
notably lower than the other models, but outperformed them at day-ahead
prediction. Another finding was the struggle to predict direct transitions
between upward and downward activations, as these events have low transition
rates. The duration-dependent Markov model is an interesting approach, and one
that is used as inspiration for the model development in this study. It captures
persistence in activation patterns, which are known to be important
\autocite{backePredictionsPricesVolumes2023a}.

\paragraph{Activation Direction Classification Models.}

Whereas imbalance forecasting estimates continuous system imbalance magnitudes,
activation-direction forecasting seeks to predict the discrete TSO decision to
activate upward, downward, or no mFRR energy. With the Nordic system's transition to
15-minute activation intervals, short-term direction forecasting has become more
important. However, the academic literature that treats direction as a primary
target remains sparse.


Svedlindh and Yngveson \autocite{svedlindhPriceFormationForecasting2025} examine
the general price formation in intraday and mFRR markets. Among other
explorations, they develop logistic regression and ANN (Artificial Neural
Network) models to make day-ahead predictions for activation direction in the
mFRR activation market. The ANN model outperforms the logistic regression,
achieving solid \textit{accuracy} and \textit{F1-scores}. They identify,
however, that \textit{class imbalance} poses a significant challenge, as
no-activation events dominate the dataset. This imbalance skews model
performance, making it difficult to accurately predict the less frequent upward
and downward activations. Despite these acknowledged challenges, Svedlindh and
Yngveson achieve promising results. They find that mFRR capacity market prices
and procured volumes are informative predictors of day-ahead activation
direction. In conclusion, they suggest that closer-to-real-time predictions could
likely provide insights to market participants.

Porras (2025) \autocite{porrasShortTermForecastingMFRR} applies an XGBoost
two-stage model to sequentially forecast activation direction and imbalance
prices at hourly resolution in SE2. The study demonstrates the potential of
tree-based methods, but it also exposes two practical limitations for
participant-oriented forecasting: (i) the model operates at hourly resolution,
and it remains unclear how well the approach would perform at the new 15-minute
resolution; and (ii) its most important predictor is “balance-direction at
\(t-0\)” that appears to be unavailable to market participants at the time of
bidding. This represents a form of feature leakage (use of variables that would
not be accessible in real decision-making) and likely inflates performance.



\subsection{Literature Synthesis}


Table \ref{tab:literature-overview} summarizes the studies reviewed in this
literature review that are most relevant to balancing-market forecasting and
uncertainty modelling. Literature is organized by authors, target variable,
temporal resolution, and uncertainty-modelling approach. 


\begin{table}[!htbp]
\centering
\caption{Overview of key forecasting and uncertainty-modelling studies}
\label{tab:literature-overview}
\begin{tabular}{p{3.2cm} p{3.2cm} p{1.8cm} p{4.0cm}}
\toprule
\textbf{Study} & \textbf{Target Variable} & \textbf{Resolution} &
\textbf{Uncertainty Model} \\
\midrule
\autocite{singhNovelDynamicImbalance2025} (2025) & Imbalance volume  & 15-min & Point forecast (regression)
\\
\midrule
\autocite{azarangMachineLearningBasedPrediction} (2025) & mFRR activation volume & 1-hour & ML point forecast
(LSTM) \\
\midrule
\autocite{backePredictionsPricesVolumes2023}\ (KoBas) (2023) & Balancing volume & 1-hour & Probabilistic LSTM;
implicit direction \\
\midrule
\autocite{plakasPredictionImbalancePrices2025}.\ (2025) & Imbalance volume and price & 1-hour & Probabilistic
(quantile regression) \\
\autocite{bankeforsDayaheadModellingElectricity} (2024) & Signed activation volume & 1-hour & Linear ML time-series;
implicit direction \\
%\midrule
%Håberg \& Doorman (2017) & Imbalance scenarios; reserve needs & 40-min horizon &
%Scenario-based uncertainty (discrete imbalance trajectories) \\
\midrule
\autocite{irrmannAnalysisModellingBalancing2023} (2023) & Activation direction and volumes & 1-hour & Expected activation ratios;
sampled regulation states \\
\midrule
\autocite{papavasiliouMultiAreaReserveDimensioning2022}\ (2022) & Reserve dimensioning & Scenario-based & Chance
constraints on imbalance scenarios \\
\midrule
\autocite{RiskConstrainedTrading} (2018) & System length (direction) & 1-hour & Probabilistic logistic
model with risk constraints \\
\midrule
\autocite{pavicTightRobustFormulation2021}\ (2023) & Reserve activation (RA ratio) & 30-min (FCR/aFRR) &
Activation ranges / bounded uncertainty sets \\
\midrule
\autocite{klaeboeDayaheadMarketBidding2022}\ (2015) & Balancing state
(up/down/none) & 1-hour & Markov transition probabilities
(duration/hour-specific) \\
\midrule
\autocite{svedlindhPriceFormationForecasting2025} (2025) & Activation direction (day-ahead) & 1-hour &
Classification (logistic, ANN) \\
\midrule
\autocite{porrasShortTermForecastingMFRR} (2025) & Activation direction and price & 1-hour & Classification
(XGBoost); sequential predictive model \\
\bottomrule
\end{tabular}
\end{table}

In summary, regressor-based approaches show promise for explicit direction
forecasting, but current studies leave questions yet to be answered: Can
direction be predicted reliably at 15-minute resolution with only available
information? And can sparse up/down events be predicted with useful precision?
The present study addresses these questions by evaluating multiple classifier
families under strict participant-feasible information constraints and at the
updated 15-minute resolution.

\textbf{Should I remove this as it is kind of research-gappy?}

\newpage
\subsection{Research Gap}
Despite notable progress and coverage of balancing-market forecasting, a couple
of gaps emerge from the literature. The simplest observation is that most
studies were conducted before the Nordic system transitioned to a 15-minute mFRR
market time unit (MTU) in March 2025. As discussed, this change alters the market dynamics and
activation patterns. While hourly forecasts remain relevant, it remains unclear
how well existing models perform at the higher resolution.

The literature focuses predominantly on continuous imbalance volumes or prices
as target variables. These quantities are of great importance, but they are
conditional on activation direction, which is not modelled explicitly in most
studies. Results are therefore affected by directional uncertainty, making
predictions noisier. In addition, activation data are often zero-inflated, so a
single direct regression model may be biased toward predicting near-zero values.
This study therefore argues that explicit direction modelling is useful for
select market participants on its own, and that it provides a natural foundation
for subsequent conditional forecasting of continuous activation quantities or
other types of analysis and decision-makiing processes.



A further consideration is that not all studies robustly address the data
availability constraints faced by market participants. Some studies incorporate
TSO-only data, while others do not explicitly evaluate whether their chosen
features would be accessible to market participants at decision time. Porras'
study \autocite{porrasShortTermForecastingMFRR}, for instance, predicts
activation direction directly, but relies on not-yet-available features, likely
inflating performance. This study emphasizes strict adherence to 
participant-feasible information sets to ensure practical relevance.

This study aims to fill these gaps by systematically evaluating machine
learning-based activation-direction forecasting in the NO1 bidding zone at
15-minute resolution, using only features available to market participants. 
