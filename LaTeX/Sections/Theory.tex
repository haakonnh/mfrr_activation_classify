\newpage
\section{Theory}
\subsection{Electricity Balancing Market}

In the Nordics, the electricity balancing markets are part of a joint
Transmission System Operator (TSO) project called the Nordic Balancing Model
(NBM) \autocite{Nordicbalancingmodel}. These markets, operated by the respective
TSOs (Statnett in Norway), are designed to ensure the stability of the power
grid by managing supply and demand imbalances in real-time
\autocite{WholesaleMarketTimeframes}. These markets facilitate the procurement
of balancing services, which are necessary to maintain the equilibrium between
electricity supply and demand. 


\subsubsection{Nordic Balancing Market Structure}
In the NBM, the balancing hierarchy consists of several layers, each serving a
specific purpose in maintaining grid stability. Figure
\ref{fig:nordic_balancing_hierarchy} illustrates the market products and
their required activation times. The fastest reserves are at the top of the hierarchy,
with slower reserves at the bottom. 

\begin{figure}[H]
    \centering
    \includegraphics[width=0.8\textwidth]{Images/TimeProducts.drawio.png}
    \caption{The Nordic balancing market response time requirement hierarchy.}
    \label{fig:nordic_balancing_hierarchy}
\end{figure}

Frequency Containment Reserves (FCR) were designed to be the first line of
defense against frequency deviations in the power grid. These reserves are
activated automatically and respond quickly to counteract sudden imbalances
between supply and demand. FCR is further divided into two categories: FCR-N
and FCR-D. FCR-D is specifically intended to address frequency deviations caused by
disturbances in the distribution network, while FCR-N focuses on normal
operating conditions in the transmission network. FCR-D should, therefore, be able
to respond faster than FCR-N to effectively manage these disturbances. 

The Fast Frequency Reserves (FFR) reserve market
was implemented in the Nordics in May 2020. These reserves are designed to
respond even more rapidly than FCR, ideally in the span of a single second. The
need for FFR arises from the increasing penetration of renewable energy sources.
Wind power is, for instance, not connected synchronously to the grid, leading to
a reduction in system inertia. Lower inertia means that frequency deviations 
occur more rapidly, necessitating faster-acting reserves like FFR to maintain
grid stability. The fast power response provided by FFR is usually sustained
for a short duration, stabilizing the frequency slightly before FCR-D takes over 
\autocite{RaskeFrekvensreserverFFR2025}. Figure \ref{fig:entsoe_balancing_philosophy}
illustrates roughly the activation times and the interplay between the different
reserve types in the Nordic balancing market. 

\begin{figure}[H]
    \centering
    \includegraphics[width=0.8\textwidth]{Images/reserve_types_graph.png}
    \caption[Illustration of the different reserve types and their activation times.]{Illustration of the different reserve types and their activation
    times \autocite{entso-eNoricBalancingPhilosophy2024}.}
    \label{fig:entsoe_balancing_philosophy}
\end{figure}

After FFR and FCR has stabilized the frequency, something must bring it back
to its nominal level. Automatic Frequency Restoration Reserves (aFRR) holds
this responsibility. These reserves are often kept activated for a couple
of minutes to ensure that the frequency is restored to its normal operating
level. Manual Frequency Restoration Reserves (mFRR) then relieves aFRR 
and maintains the balance until normal operations are restored.

\textbf{mFRR theory?}

\subsubsection{Reserve Market Concepts}
Reserve markets are platforms where market participants can offer their
balancing services to the grid operator. These markets operate on the principle
of supply and demand, where participants can bid to provide reserves at
specific prices. The markets are then cleared based on the bids received, 
ensuring that the most cost-effective resources are utilized to maintain grid 
stability. Reserve markets can be broadly categorized into two types:
activation markets and capacity markets. Only aFRR and mFRR markets will be 
discussed further, as they are the most relevant for this study.



\paragraph{Capacity Markets (CM).}
Capacity is a market mechanism that ensures the availability of sufficient
resources to maintain grid stability and reliability. Capacities are procured
prior to real-time operation to guarantee the availability of balancing
resources at the time of operation
\autocite{entso-eNoricBalancingPhilosophy2024}. Balance Service Providers (BSP),
market participants that provide balancing services
\autocite{BalancingServiceProvider2024}, can offer their capacities to the grid
operator through the Nordic aFRR capacity market or in the national mFRR
capacity markets. Capacity market participants are compensated for making their
resources available to the grid operator, regardless of whether their resources
are activated or not. They are, however, obligated to deliver the offered
capacity when called upon by the grid operator.

\paragraph{Energy Activation Markets (EAM).}\label{par:energy-activation-markets}
Activation markets operate closer to real-time and are designed to procure
balancing energy to address immediate imbalances in the power grid. In the Nordic
region, the aFRR and mFRR activation markets serve this purpose. In the 
mFRR EAM in Norway, BSPs submit bids to Statnett at least 45 minutes before
the activation period. These bids specify the amount of up- or down-regulation
capacity the BSP is willing to provide and the corresponding price. Statnett
then forwards the bids to the clearing algorithm Nordic Libra AOF, which
provides the activation volumes for the operational quarter-hour. Statnett
then activates the selected bids based on the activation volumes provided
by Nordic Libra AOF \autocite{statnettVilkarMFRRAktiveringsmarked2024}.


%\paragraph{Imbalance Price.} The imbalance price is the price used for
%settlement in the balancing market, reflecting the cost of correcting imbalances
%in the power system caused by discrepencies between supply and demand. A single
%imbalance price is calculated for a given imbalance period. The price is thus
%dependent on the imbalance direction, either up-regulation, down-regulation, or
%no activation. For up- and down-regulation, the imbalance price is determined
%based on the up- and down-regulation prices, respectively, or the
%volume-weighted aFRR energy up- and down-regulation prices if applicable. If
%there is no activation, a Value of Avoided Activation (VoAA) along with the
%incentivizing component (IC) is used
%\autocite{esettNordicImbalanceSettlement2024}.


%\textbf{I am not quite steady on imbalance price, is this actually only used 
%to settle imbalances for balance responsible parties (BRP) and not BSPs in
%the mFRR market?}



\subsection{Machine Learning Theory}

Machine learning (ML) is a subset of artificial intelligence (AI) that focuses
on developing algorithms and statistical models that enable computers to perform
specific tasks without explicit instructions. Instead, these algorithms learn
from data, identifying patterns and making decisions based on the information
provided. 

\subsubsection{Supervised Learning for Time-Dependent Classification}
Supervised learning is a machine learning paradigm where models are trained to
map input data to specific outputs based on example input-output pairs. In this
project, the input data consists of information about the current state relevant
to the occurence mFRR activations.
The output is a ternary label indicating
whether an up-regulation, down-regulation, or no activation occurs. Thus, the
output variable is categorical, making it a classification task, where the
goal is to assign input data points to one of several predefined classes.

Supervised learning is a machine learning paradigm in which models are trained
to learn a mapping from input variables to target outputs based on labeled
example pairs. When the target variable takes values from a set of predefined
discrete categories, the task is formulated as a classification problem, where
the objective is to assign each input instance to one of the several predefined
classes. However, standard supervised learning is not applicable in this
project, as the data is ordered and time-dependent. This is often referred to as
\textit{event prediction}, where \textit{events} are defined as nontrivial
occurrences in specific locations and time
\autocite{zhaoEventPredictionBig2022}. 

Chronological structure has wideranging implications for model training and
evaluation. Standard supervised learning techniques often assume that data
points are independent and identically distributed (i.i.d.), which is not the
case for time-dependent data. Temporal dependencies must be accounted for, as
past events can influence future outcomes. This calls for specialized model and
data engineering techniques that can effectively capture and leverage these
relationships. Perhaps most importantly, the evaluation methodology must respect
the temporal order of the data to avoid data leakage and ensure that the model's
performance is assessed in a realistic manner. The model would, for instance,
produce unrealistic results if it has access to future data points when making
predictions for a given time step.

\subsubsection{Class Imbalance}\label{sec:theory-class-imbalance}
Class imbalance is a common challenge in classification tasks, particularly in
real-world applications where certain classes occur much more frequently than
others \autocite{ClassImbalanceProblem}. This class imbalance
presents a significant challenge for model training and evaluation. Most ML
algorithms are designed to optimize overall accuracy, which can lead to models
that are biased towards the majority class, since correctly predicting the
majority class contributes more to overall performance. If the minority class is
important, special considerations must
be taken to ensure that the model learns to effectively recognize and predict
these rare events.

\subsubsection{Tree-Based Models}

Given the characteristics of the problem, tree-based machine learning models are
well suited for activation direction prediction. Such models can capture
nonlinear relationships and complex interactions between features without
requiring strong parametric assumptions. They are also robust to differences in
feature scale and can naturally accommodate heterogeneous inputs, including
prices, flows, forecasts, and lagged indicators.

Importantly, tree-based models integrate effectively with feature-engineering
approaches that encode temporal information through lagged values and
persistence measures. Rather than modeling sequential dynamics explicitly, these
models infer temporal structure indirectly from engineered features that
summarize recent system behavior. This makes them particularly suitable for
settings where direct access to high-resolution real-time data is limited.


