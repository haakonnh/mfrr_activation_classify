\section{Theory}
\subsection{Electricity Balancing Market}

Electricity balancing markets are mechanisms designed to ensure the stability
and reliability of the power grid by managing supply and demand in real-time.
These markets facilitate the procurement of balancing services, which are
necessary to maintain the equilibrium between electricity supply and demand.
Balancing markets operate on the principle of economic efficiency, where
market participants can offer their flexibility to the grid operator in
exchange for compensation.


\subsubsection{Nordic Balancing Market Structure}
In Nordic countries, the balancing hierarchy consists of several layers,
each serving a specific purpose in maintaining grid stability. Figure
\ref{fig:nordic_balancing_hierarchy} illustrates the different reserve types, 
their activation times, and their place in the hierarchy.

\textbf{Not manual anymore}
\begin{figure}[H]
    \centering
    \includegraphics[width=0.8\textwidth]{Images/reserves_hierarchy_entsoe.png}
    \caption[The Nordic balancing market hierarchy, illustrating the different reserve types and their activation times.]{The Nordic balancing market hierarchy, illustrating the different
    reserve types and their activation times \autocite{entso-eNoricBalancingPhilosophy2024}.}
    \label{fig:nordic_balancing_hierarchy}
\end{figure}

Frequency Containment Reserves (FCR) were designed to be the first line of
defense against frequency deviations in the power grid. These reserves are
activated automatically and respond quickly to counteract sudden imbalances
between supply and demand. FCR is further divided into two categories: FCR-N
and FCR-D. FCR-D is specifically intended to address frequency deviations caused by
disturbances in the distribution network, while FCR-N focuses on normal
operating conditions in the transmission network. FCR-D should, therefore, be able
to respond faster than FCR-N to effectively manage these disturbances. 

The Fast Frequency Reserves (FFR) reserve market
was implemented in the Nordics in May 2020. These reserves are designed to
respond even more rapidly than FCR, ideally in the span of a single second. The
need for FFR arises from the increasing penetration of renewable energy sources.
Wind power is, for instance, not connected synchronously to the grid, leading to
a reduction in system inertia. Lower inertia means that frequency deviations 
occur more rapidly, necessitating faster-acting reserves like FFR to maintain
grid stability. The fast power response provided by FFR is usually sustained
for a short duration, stabilizing the frequency slightly before FCR-D takes over 
\autocite{RaskeFrekvensreserverFFR2025}. Figure \ref{fig:entsoe_balancing_philosophy}
illustrates roughly the activation times and the interplay between the different
reserve types in the Nordic balancing market. 

\begin{figure}[H]
    \centering
    \includegraphics[width=0.8\textwidth]{Images/reserve_types_graph.png}
    \caption[Illustration of the different reserve types and their activation times.]{Illustration of the different reserve types and their activation
    times \autocite{entso-eNoricBalancingPhilosophy2024}.}
    \label{fig:entsoe_balancing_philosophy}
\end{figure}

After FFR and FCR has stabilized the frequency, something must bring it back
to its nominal level. Automatic Frequency Restoration Reserves (aFRR) holds
this responsibility. These reserves are often kept activated for a couple
of minutes to ensure that the frequency is restored to its normal operating
level. Manual Frequency Restoration Reserves (mFRR) then relieves aFRR 
and maintains the balance until normal operations are restored.

\subsubsection{Reserve Market Concepts}
Reserve markets are platforms where market participants can offer their
balancing services to the grid operator. These markets operate on the principle
of supply and demand, where participants can bid to provide reserves at
specific prices. The markets are then cleared based on the bids received, 
ensuring that the most cost-effective resources are utilized to maintain grid 
stability. Reserve markets can be broadly categorized into two types:
activation markets and capacity markets. Only aFRR and mFRR markets will be 
discussed further, as they are the most relevant for this study.



\paragraph{Capacity Markets}
Capacity is a market mechanism that ensures the availability of sufficient
resources to maintain grid stability and reliability. Capacities are 
procured prior to real-time operation to guarantee the availability of
balancing resources at the time of operation
\autocite{entso-eNoricBalancingPhilosophy2024}. BSPs can offer their capacities 
to the grid operator through the Nordic aFRR capacity market or in the 
national (Statnett in Norway) mFRR capacity markets. Capacity market 
participants are compensated for making their resources available to the
grid operator, regardless of whether their resources are activated or not.
They are, however, obligated to deliver the offered capacity when called upon
by the grid operator.

\paragraph{Energy Activation Markets (EAMs)}
Activation markets operate closer to real-time and are designed to procure
balancing energy to address immediate imbalances in the power grid. In the Nordic
region, the aFRR and mFRR activation markets serve this purpose. In the 
mFRR EAM in Norway, BSPs submit bids to Statnett at least 45 minutes before
the activation period. These bids specify the amount of up- or down-regulation
capacity the BSP is willing to provide and the corresponding price. Statnett
then forwards the bids to the clearing algorithm Nordic Libra AOF, which
provides the activation volumes for the operational quarter-hour. Statnett
then activates the selected bids based on the activation volumes provided
by Nordic Libra AOF \autocite{statnettVilkarMFRRAktiveringsmarked2024}.

The mFRR EAM underwent a significant reform March 4th 2025, transitioning
from national mFRR activation systems to a Nordic-wide mFRR activation market.
The previously hourly resolution was also changed to a quarter-hourly resolution.
This resolution change allows for more precise balancing, introducing more
nuanced price signals. 

\paragraph{Imbalance Prices}




\paragraph{mFRR Market in Norway (will refactor this or remove
this)}
In Norway, the mFRR market plays a crucial role in maintaining the
stability of the power grid. Market participants can offer their
flexibility to the grid operator through both activation and capacity
markets. 

\subsection{Machine Learning Theory}

Machine learning (ML) is a subset of artificial intelligence (AI) that focuses
on developing algorithms and statistical models that enable computers to perform
specific tasks without explicit instructions. Instead, these algorithms learn
from data, identifying patterns and making decisions based on the information
provided. In this project, machine learning techniques are employed to predict the
occurrence of mFRR activations in NO1 based on historical data and relevant system 
state indicators.

\subsubsection{Supervised Learning for Time-Dependent Classification}
Supervised learning is a machine learning paradigm where models are trained
to map input data to specific outputs based on example input-output pairs. In 
this project, the input data consists of information about the current state
relevant to the occurence mFRR activations. The output is a ternary
label indicating whether an up-regulation, down-regulation, or no activation
occurs. Thus, the output variable is categorical, making this a classification task,
where the goal is to assign input data points to one of several predefined classes.
However, the problem at hand is not a standard supervised learning task, as
the data is ordered and time-dependent. This is often referred to as \textit{event
prediction}, where \textit{events} are defined as nontrivial occurrences in specific 
locations and time \autocite{zhaoEventPredictionBig2022}. In this project, location 
is fixed to the NO1 bidding zone, while time is discretized into quarter-hourly
intervals. 

The chronological structure has wideranging implications for model training and
evaluation. Standard supervised learning techniques often assume that data points
are independent and identically distributed (i.i.d.), which is not the case
for time-dependent data. Temporal dependencies must be accounted for, as past events can
influence future outcomes. This calls for specialized model and data engineering
techniques that can effectively capture and leverage these relationships.
Perhaps most importantly, the evaluation methodology must respect the temporal order
of the data to avoid data leakage and ensure that the model's performance is assessed
in a realistic manner. The model would, for instance, produce over-optimistic results 
if it has access to future data points when making predictions for a given time step.

\subsubsection{Class Imbalance}
\textbf{Perhaps not introduce this yet? It is quite an important topic, though, 
as it introduces one of the biggest challenges.}
mFRR up activations are particularly rare events in the NO1 bidding zone compared 
to down activations and non-activations. This \textit{class imbalance} presents a 
significant challenge for model training and evaluation. Most ML algorithms are designed
to optimize overall accuracy, which can lead to models that are biased towards the
majority class, since correctly predicting the majority class contributes more to 
overall performance. If the minority class is important, as is the case with mFRR up 
activations, special considerations must be taken to ensure that the model learns to
effectively recognize and predict these rare events.

\subsubsection{Tree-Based Models}

\paragraph{Random Forest / Extra Trees}

\paragraph{Gradient Boosting}

\textbf{Metrics introduction?}
