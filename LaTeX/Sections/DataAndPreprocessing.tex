\newpage
\section{Methodology}
\subsection{Data Sources and Preprocessing}\label{sec:data-and-preprocessing}


The gathered data spans a period from January 1, 2024, to December 1, 2025, 
providing a comprehensive view of the mFRR activation patterns over nearly two
years. This extensive dataset allows for robust model training and evaluation,
capturing seasonal variations and other temporal dynamics in the electricity
market.

\subsubsection{Data Sources}

\subsubsection{Data Coverage and Time Span}

\subsubsection{Nord Pool Data}
Nord Pool is the leading power market in Europe, facilitating the trading of
electricity across multiple countries. The data from Nord Pool includes
information on market prices, trading volumes, and other relevant metrics
that can influence mFRR activations. Data was obtained manually through the Nord 
Pool data portal as CSV files and then combined into a single dataset for analysis
\autocite{PowerMarketData}.
This tedious manual approach was necessary due to the Nord Pool API not being 
available for this project. Data API access requires a commercial agreement with
Nord Pool, which was not feasible within the project's scope. This is not an 
issue for this project and research as real-time data is not required for
training and evaluating the models. If the models were to be deployed in a
real-time setting, however, access to real-time data through the API would be 
essential. The following subsections describe the specific datasets obtained
from Nord Pool.


\paragraph{mFRR activation data}
The primary dataset used in this study consists of manual Frequency
Restoration Reserves (mFRR) activation data from the Nordic electricity
market, specifically for the bidding zone NO1. This data includes accepted 
and activated up- and down-regulation bids at a 15-minute resolution. The 
activated volumes provide the target variable for the prediction models, indicating
whether an mFRR activation occurred in a given 15-minute interval. Figure 
\ref{fig:mFRR_activation_distribution} displays the distribution of mFRR activations
over the dataset period. The figure especially highlights the infrequent nature 
of up-activations, which occur far less often than down-activations. This 
phenomenon induces a \textit{class imbalance} in the prediction task, which 
in general makes it more challenging for models to accurately predict the minority
class \autocite{ClassImbalanceProblem}. 

\begin{figure}[H]
    \centering
    \includegraphics[width=\textwidth]{Images/mfrr_distribution.png}
    \caption{mFRR activation distribution.}
    \label{fig:mFRR_activation_distribution}
\end{figure}

The distribution also reveals the majority class with 56,9\% of
all intervals having no activation. Activations occur in 43,1\% of the intervals,
with down-activations being the most common at 32,5\% and up-activations
being the least common at 10,6\%. Thus, if one were to consider the binary case
of activation vs. no activation, the classes would be approximately balanced.
Distinguishing between up- and down-activations, however, makes the problem
more nuanced and challenging.


\paragraph{Cross-zonal flows}
Cross-zonal flows refer to the electricity flows between different
bidding zones in the Nordic market. In this project, only cross-zonal flows
involving the NO1 bidding zone are considered: flows between NO1 and SE3, NO1 and
NO3, NO1 and NO5, and NO1 and SE3. These flows are crucial for maintaining
grid stability and optimizing the use of available resources. The dataset
includes information on cross-zonal flows to provide additional context for
mFRR activations. Figure \ref{fig:cross_zonal_flows_dist} shows the distribution of cross-zonal flow
directions for the NO1 bidding zone. The figure indicates that flows between NO1
and NO3, NO5, and SE3 are drastically skewed towards imports into NO1, while flows 
between NO1 and NO2 are reversely skewed. NO1-NO3 and NO1-NO5 show the most
pronounced skewness, with very few occurrences of exports from NO1 to these zones.
\textbf{This might be interesting to talk about later, as imbalance here may
make the directionality more predictive.}

\begin{figure}[H]
    \centering
    \includegraphics[width=\textwidth]{Images/cross_zonal_flows_distribution_bar.png}
    \caption{Cross-zonal flow distributions for the NO1 bidding zone.}
    \label{fig:cross_zonal_flows_dist}
\end{figure}

Figure \ref{fig:cross_zonal_imports} and \ref{fig:cross_zonal_exports} show the 
relative utilization of the cross-zonal connections for NO1 as density plots. The
figures illustrate how heavily the connections are utilized for imports and exports,
respectively. The utilization is calculated as the ratio between actual flow and
the NTC (Net Transfer Capacity) capacity of the connection. The export utilization
figure highlights the lack of exports from NO1, except for the NO2 connection, 
indicated by the tail thickness between 0.4 and 1.0 utilization. The import
utilization figure, on the other hand, shows that all connections, except NO2-NO1,
are heavily utilized for imports, with many thick tails approaching full utilization.
The NO1-NO2 connection is thus almost exclusively used for exports from NO1 to NO2, 
whilst the other connections are primarily used for imports into NO1.

\begin{figure}[H]
    \centering
    \begin{subfigure}[t]{0.49\textwidth}
        \centering
        \includegraphics[width=\linewidth]{Images/import_utilization.png}
        \caption{Import utilization.}
        \label{fig:cross_zonal_imports}
    \end{subfigure}
    \hfill
    \begin{subfigure}[t]{0.49\textwidth}
        \centering
        \includegraphics[width=\linewidth]{Images/export_utilization.png}
        \caption{Export utilization.}
        \label{fig:cross_zonal_exports}
    \end{subfigure}
    \caption{Cross-zonal flow utilizations for the NO1 bidding zone calculated
    as the ratio between actual flow and NTC capacity.}
    \label{fig:cross_zonal_flows_side_by_side}
\end{figure}


\paragraph{Load and production data}
Load and production forecasts provide insights into the expected
system state. Forecast may on their own provide valuable information about
potential mFRR activations, but when combined with actual load and production
data, the model can learn to identify discrepancies between expected and
actual system states. Such discrepancies often lead to imbalances that require
mFRR activations to restore balance. The different production sources 
(e.g., hydro, wind, thermal) have varying characteristics and impacts on grid
stability. Among them, wind power is particularly relevant due to its
intermittent nature, which can lead to sudden changes in generation levels.
Wind power production data is therefore predicted to have the biggest impact
on mFRR activations among the different production types.

\begin{table}[ht]
\centering
\caption{Summary statistics for wind-related features (2024--2025, NO1)}
\begin{tabular}{lrrrrrrrr}
\toprule
Metric & Mean & Std & Min & P10 & P50 & P90 & Max & Count \\
\midrule
Wind DA Forecast        & 121.64 & 96.34 & 0.0   & 16.0   & 95.0   & 274.0  & 370.0  & 62,680 \\
Wind Intraday Forecast  & 135.80 & 104.55 & 0.0  & 15.0   & 111.0  & 294.0  & 376.0  & 38,972 \\
Wind Actual Production  & 120.38 & 102.72 & 0.0  & 8.0    & 91.0   & 283.0  & 380.0  & 65,568 \\
Wind Revision (ID--DA)  & 12.59  & 15.27  & 0.0  & 1.0    & 7.0    & 30.0   & 151.0  & 38,972 \\
DA--Actual Error        & 0.03   & 38.85  & -250.0& -44.0  & -2.0   & 47.0   & 221.0  & 62,680 \\
ID--Actual Error        & 2.97   & 35.30  & -227.0& -38.0  & 1.0    & 46.0   & 189.7  & 38,972 \\
Abs DA Error (\%)       & 0.58   & 0.99   & 0.0  & 0.038  & 0.24   & 1.36   & 5.0    & 61,486 \\
Abs ID Error (\%)       & 0.42   & 0.75   & 0.0  & 0.032  & 0.19   & 0.92   & 5.0    & 38,370 \\
Wind Share              & 0.054  & 0.047  & 0.0  & 0.0036 & 0.040  & 0.128  & 0.228  & 65,568 \\
\bottomrule
\end{tabular}
\label{tab:wind_feature_stats}
\end{table}

Load/consumption data is much simpler in nature, as \textit{who} or \textit{what},
essentially the source of consumption, is not as relevant as the source of production.
Consumption forecasts and actual consumption data can still be useful, however,
as sudden changes in consumption patterns can lead to imbalances that
require mFRR activations. 

\subsubsection{NUCS}
NUCS, or the Nordic Unavailability Collection System, is a service for
collection of data on unavailable data in the Nordic power system. NUCS
is an important part of this project, as it provides otherwise unavailable
data that served as features in the models. NUCS is unique from the other 
data sources used in this project, as it provides data through an API (Application
Programming Interface) 
\autocite{StaticContentNordic}. This allows for automated data retrieval, which is
especially useful for real-time applications. However, as this project does
not have access to comprehensive real-time data, the NUCS API was only used
to gather historical data for the training and evaluation of the models.

\paragraph{aFRR procurement prices}
aFRR procurement data is generally unavailable. However, through the NUCS API,
historical aFRR procurement prices and volumes for the NO1 bidding zone 
were obtained. This data provides insights into the costs associated with
aFRR activations and can be used as additional features in the models. 

\begin{figure}[H]
    \centering
    \includegraphics[width=\textwidth]{Images/nucs_afrr_hourly_hist.png}
    \caption{A histogram of hourly aFRR procurement prices for the NO1
bidding zone from NUCS data.}
    \label{fig:nucs_afrr_hourly_hist}
\end{figure}

\begin{figure}[H]
    \centering
    \includegraphics[width=\textwidth]{Images/nucs_afrr_hourly_prices.png}
    \caption{Hourly aFRR procurement prices for the NO1 bidding zone from NUCS data.}
    \label{fig:nucs_afrr_hourly_prices}
\end{figure}

Figure \ref{fig:nucs_afrr_hourly_hist} and \ref{fig:nucs_afrr_hourly_prices} show
a histogram and time series plot of the hourly NO1 aFRR procurement prices between
January 1, 2024, and \lastDate. Both figures highlight a problem in the data
before July 2024, where there are many missing values represented as zeros.
The data appears complete after this date. Intermittent missing values can be
handled during preprocessing in multiple ways. In time series data, simply removing
the rows with missing values is not an option, as it would break the time continuity.
One approach is to use interpolation to estimate the missing values based on
the surrounding data points. Another approach is to use forward or backward filling, 
where missing values are filled with the last known value or the next known value,
respectively. The choice of method depends on the nature of the data and the
specific requirements of the analysis or model being used.

\subsubsection{ENTSO-E}
ENTSO-E, the European Network of Transmission System Operators for Electricity,
is a key organization in the European electricity market. ENTSO-E provides
a wide range of data related to electricity generation, consumption, and grid
operations across Europe \autocite{DataStandardisation}.

\paragraph{aFRR Activation Data}
aFRR prices and capacities are available through NUCS as discussed earlier, but
aFRR activation data is not. However, ENTSO-E provides detailed data on aFRR 
regulations via their "Accepted Offers and Activated Balancing Reserves" dataset.
This dataset is supposed to provide information about up and down regulations from 
all balancing markets. Only aFRR data seems to be available, however, which is
sufficient for this project. The data is available at an hourly resolution, which
is coarser than the 15-minute resolution of the mFRR data. This data is, in the same
manner as other 1-hour resolution data, resampled with forward filling to match the
15-minute resolution of the main dataset.

\paragraph{Crucial: Time restriction. (Probably move parts of this to somewhere else
as these concepts are important.)} aFRR activation data is naturally not
available at the target time $t+4$ and prediction time $t$ as these 
activations has not yet occurred or finished. The 1-hour resolution is 
problematic as it essentially causes the data to become available in clusters
of four 15-minute intervals at a time. This causes inconsistencies in the 
what specific lag features are available at different prediction times. For 
instance, assume data becomes available immediately after the hour. Then, at
time $t$ = 10:00, the last available aFRR activation data would be from
09:00-10:00. This means that lag features for $t-1$, $t-2$, $t-3$, and $t-4$
would be available. However, in the worst case, at a prediction time $t$ = 10:45
, the last available aFRR activation data would still be from 09:00-10:00. This means
that only lag features starting at $t-4$ would be available, while $t-1$, $t-2$,
and $t-3$ would not. This inconsistency in available lag features complicates the
model training and evaluation process. There are multiple ways to handle this
issue:

\begin{itemize}
    \item \textbf{Option 1:} Only use lag features that are always available,
    i.e., only use lag features starting from $t-4$ and further back. This is
    the easiest option to implement but it sacrifices potentially valuable
    information from more recent time steps.

    \item \textbf{Option 2:} Create separate models for different prediction
    times within the hour (e.g., one model for predictions at 00, 15, 30,
    and 45 minutes past the hour). Each model would be trained with the
    each with their own set of available lag features. This approach maximizes
    the use of available data but requires maintaining multiple models. 
    The more gathered datasets are of a 1-hour resolution, the more valuable
    this approach becomes.

    \item \textbf{Option 3:} Use imputation techniques to estimate the missing
    lag features based on available data. This approach allows for a single
    model to be used but introduces uncertainty due to the imputation process.
    
\end{itemize}

\subsubsection{Dataset structure}\label{sec:dataset_structure}
The data is represented as a time series, where each record in the dataset
consists of a set of attributes connected to one point in time. More
specifically, the data contains a sequence of 15-minute interval time
stamps. Each time stamp may or may not have an associated activation, which
is the target variable the model is trying to predict. The features
describe the system state at that time stamp, providing context for the
model to learn from.


\subsubsection{Restrictions}\label{sec:restrictions}
In an ideal world, a market participant would be able to choose to bid or
not in real time. The mFRR activation market, however, is not ideal in
this sense. Bids are accepted for time slots $t+4$ intervals into the
future, i.e. one hour into the future. This greatly restricts market
actors in terms of how fast they can act on recent information. In fact,
at the time of bid closing for a time slot $t+4$, only mFRR activation
data from $t-3$ and earlier is available. Thus there are seven 15-minute
intervals of unavailable data. This data would be the most
useful for the model, since activation data at time $t$ correlates with
the system state and activations close to $t$.

It would, for instance, be considerably easier to identify streaks of
activations, as activations often occur in long contiguous sequences. In the real
world scenario, data from $t-3$ and before is still useful, but can often be
misleading and cannot be trusted in isolation. The dataset developed in
this project investigates ways to provide the model with crucial context
about the system's state of stress. The intuition is that data such as
physical cross-zonal flows in and out of NO1 may correlate with NO1 mFRR
activations. When many such features are combined, the model may develop
complex relationships between them, enhancing its predictive performance.

Most of the available and useful data are not available in real time. For
instance, consumption and production data are often published with a delay of
several hours. The model must therefore rely on features that are available
in real time, or with minimal delay. Such features often take the form of
forecasts, which are available ahead of time. Forecasts are inherently
uncertain, but they still provide valuable information about the expected
system state.

\subsubsection{Class Imbalance}\label{sec:class_imbalance}
