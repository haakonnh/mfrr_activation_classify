\newpage
\section*{Abstract}
The Nordic power system's transition to automated manual Frequency Restoration
Reserve (mFRR) activation at a 15-minute resolution on 4 March 2025 represents a
structural shift in balancing market operations and increases the value of
short-term forecasting for market participants. While existing literature has
largely focused on forecasting activation volumes or prices, often from a
transmission system operator perspective, less attention has been paid to
predicting mFRR activation direction under the information constraints faced by
market participants. For demand-side aggregators of flexible resources,
activation direction (up-regulation, down-regulation, or no activation)
constitutes the most critical short-term signal, as it determines the likelihood
of activation independently of price or volume magnitude.

This study investigates the feasibility of short-term forecasting of mFRR energy
activation direction at a 15-minute resolution using only data available to
market participants at gate closure. Focusing on the Norwegian NO1 bidding
zone, a supervised multi-class classification framework is developed using
extensive feature engineering across market prices, cross-zonal flows,
production and load forecasts, and temporal persistence indicators. Tree-based
machine-learning models are developed and tuned using the open-source AutoML
framework AutoGluon, and evaluated against a persistence-based naïve baseline
using temporally consistent training, validation, and test splits that respect
real-time information availability.

The results demonstrate that the majority of predictive power for mFRR
activation direction arises from short-term temporal activation persistence. The
findings indicate that explicit direction forecasting is both practically
relevant for participant decision-making as correctly predicted activation
directions entail meaningful economic value, and methodologically valuable as an
intermediate step toward conditional forecasting of activation volumes and
prices.

By providing one of the first participant-feasible, 15-minute resolution studies
of mFRR activation direction under the new Nordic market design, this work
contributes empirical evidence on short-term predictability in automated
balancing markets and establishes a foundation for future conditional and
probabilistic activation forecasting frameworks.