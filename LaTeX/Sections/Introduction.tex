\section{Introduction}
\subsection{Background}
The increasing share of weather-dependent renewable energy sources has reduced
the inertia of modern power systems and made short-term system imbalances more 
frequent and difficult to predict \autocite{caiLowInertiaPower2023}. To maintain
frequency quality, Nordic transmision system operators (TSO) rely on on various
balancing resources, among which manual Frequency Restoration Reserves (mFRR) play
a key role. The mFRR energy activation market (mFRR EAM) enables market participants
to offer up- and down-regulation reserves that the TSO can activate when sustained
imbalances occur \autocite{entso-eNoricBalancingPhilosophy2024}.

Historically, mFRR activations were manual and scheduled on an hourly basis.
On 4 March 2025, the Nordic region introduced a new automatic mFRR activation
platform with a 15-minute resolution \autocite{ConfirmationMFRREAM2025}. This reform
aligns the Nordic system with European balancing-market standards and allows
activation signals to more closely follow real-time system conditions. Automated
activation of mFRR reserves enables faster response times, making it possible 
to adress quarter-hourly imbalances effectively \autocite{NordicPowerMarket}.



\begin{figure}[H]
    \includegraphics[width=0.8\textwidth]{Images/dnv_electricity_supply_by_source.png}
    \centering
    \caption{Norway electricity supply by power station and net imports with 
    projections to 2050 \
    \autocite{EnergyTransitionOutlook}}
    \label{fig:dnv-electricity-supply-by-source}
\end{figure}

\textbf{Struggling with what to include in background as I feel like much of interesting
background information kind of fits in motivation.}


\subsection{Motivation}
mFRR activations have direct economic implications for market participants.
Accurately anticipating whether up- or down-regulation is likely enables
participants to choose between committing flexible resources to capacity
markets or bidding directly into the activation market, where revenues depend
on actual activations. Capacity markets provide stable income, whereas
activation markets offer potentially higher but uncertain returns. Being able
to accurately assess the likelihood of an activation therefore supports more
informed and profitable bidding strategies.

For aggregators of distributed flexible resources, such as electric vehicles,
heat pumps, and batteries, predicting mFRR activations is particularly valuable.
When commiting flexibility to the mFRR capacity market, the aggregator earns
a fixed payment for being available to provide reserves. However, this also
means the resources are locked up and cannot be used for other purposes during
the committed period. Participation in the activation market locks resources for 
shorter periods,
allowing the aggregator to remain flexible. 
The guaranteed income from capacity payments must thus be weighed 
against the opportunity cost of missing out on potential revenues from the activation
market. The correct choice depends on the likelihood of activations occurring, which
is inherently uncertain.

A reliable activation prediction model can substantially reduce this uncertainty
by providing an predictions of whether up- or down-regulation is likely in
future intervals. With such information, an aggregator can manage its portfolio
more strategically, for example, by reserving flexibility for the activation
market only when the probability of activation is sufficiently high, or by
safely committing to the capacity market when activations appear unlikely.
Accurate predictions therefore enable more efficient utilization of distributed
resources and support more economically rational market participation.


The recent shift to 15-minute resolution further increases the relevance of such
models. While higher time resolutions offer greater responsiveness, it also
creates a shorter decision window and increases sensitivity to rapidly changing
system conditions. This motivates data-driven prediction methods as an
approach to support market bidding strategies.
An
interesting application is to use the predictions to inform reinforcement learning-based
bidding strategies for aggregators of distributed flexible resources. Reinforcement learning
agents can learn to make optimal decisions, but require a reliable model of the environment
to base their strategies on. A well-performing mFRR activation prediction model could
serve as this model of the environment, enabling the development of near-optimal
bidding strategies that adapt to predicted market conditions.
\textbf{Is this too much reinforcement learning talk for an introduction?}



\subsection{Research Question}

The central research question of this study is: How well can mFRR up- and 
down-regulation activations be predicted  at 15-minute resolution using only 
real-time-available system state features and historical activation/price data?
\textbf{Change this I suppose?}

\subsection{Outline}

This report is structured as follows: Chapter 2 provides an overview 
of the Nordic balancing markets and related work. Chapter 3 reviews existing literature 
on imbalance and activation forecasting and identifies the research gaps addressed in 
this study. Chapter 4 details the modelling methodology, including data preprocessing,
feature engineering, and model evaluation and selection. 
Chapter 5 presents the results, while Chapter 6 discusses implications, limitations, and 
directions for 
future work.