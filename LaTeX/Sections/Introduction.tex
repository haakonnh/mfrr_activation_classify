\section{Introduction}
\subsection{Background}

The increasing share of weather-dependent renewable energy sources, particularly
wind power as illustrated in Figure \ref{fig:dnv-electricity-supply-by-source},
has fundamentally altered the dynamic behavior of modern power systems. Unlike
conventional synchronous generation, these resources contribute little or no
rotational inertia, reducing the system's ability to passively resist frequency
deviations following disturbances. As a result, frequency deviations now evolve
more rapidly and with greater amplitude, making short-term system imbalances
both more frequent and more difficult to predict
\autocite{caiLowInertiaPower2023}. These challenges are further exacerbated by
forecast uncertainty in renewable generation and load, which can lead to rapid
and unforeseen deviations from scheduled production and consumption. 

To maintain frequency quality and operational security under these conditions,
Nordic transmission system operators (TSOs) rely on a hierarchy of balancing
resources designed to operate at different time scales. While fast-acting
reserves such as FCR and aFRR address immediate frequency deviations, manual
Frequency Restoration Reserves (mFRR) play a key role in correcting sustained
imbalances and restoring the system to a stable operating point. The mFRR energy
activation market (mFRR EAM) enables market participants to offer up- and
down-regulation reserves that the TSO can activate when such imbalances persist
\autocite{entso-eNoricBalancingPhilosophy2024}. 

Historically, mFRR activations in the Nordic region were manual and scheduled on
an hourly basis. Under this structure, activation signals were constrained to hourly
blocks, limiting the ability of the balancing mechanism to respond efficiently
to short-lived imbalances or rapid changes in system conditions. On March 4th
2025, the Nordic region introduced a new automatic mFRR activation platform
operating at a 15-minute resolution \autocite{statnettConfirmationMFRREAM2025}. This
reform represents a significant shift in balancing market design, aligning the
Nordic system with European standards and facilitating closer integration with
pan-European balancing initiatives.

The transition to automated 15-minute mFRR activation allows balancing actions
to more closely track real-time system dynamics, improving the precision and
timeliness of imbalance correction. Automated activation enables faster response
times and reduces structural imbalances arising from temporal aggregation
\autocite{NordicPowerMarket}. At the same time, the finer temporal resolution
fundamentally changes the decision-making environment for market participants.
Bidding decisions must now be made over shorter horizons and with increased
sensitivity to rapidly evolving system conditions, magnifying the importance of
reliable short-term forecasting and situational awareness in the mFRR energy
activation market.



\begin{figure}[H]
    \includegraphics[width=0.65\textwidth]{Images/dnv_electricity_supply_by_source.png}
    \centering
    \caption{Norway electricity supply by power station and net imports with 
    projections to 2050
    \autocite{EnergyTransitionOutlook}.}
    \label{fig:dnv-electricity-supply-by-source}
\end{figure}


\subsection{Motivation}
mFRR activations have direct economic implications for market participants.
While capacity markets provide stable income, energy activation markets offer
higher but uncertain returns. Anticipating whether mFRR up- or down-regulation
is likely therefore affects how flexible resources should be allocated between
capacity commitments and activation market participation. Being able to forecast
when no activation is likely to occur is equally important, as capacity market
revenues can then be prioritized without risking missed activation
opportunities. This trade-off is particularly relevant for aggregators managing
distributed flexible assets such as electric vehicles, heat pumps, and
batteries. Committing flexibility to the mFRR capacity market secures a fixed
payment but binds resources for the contracted period. In contrast,
participation in the activation market preserves operational flexibility, as
resources are only constrained during actual activations. Aggregators must
therefore balance guaranteed capacity revenues against the expected value of
future activation opportunities, under substantial uncertainty.

The transition to a 15-minute market time unit represents not only a structural
change in market operation but also an opportunity for more informative
short-term decision support. Finer temporal resolution allows activation signals
to better reflect rapidly evolving system conditions, creating richer and more
frequent market signals. At the same time, this increased granularity elevates
the importance of short-term forecasts, as market participants must make
decisions over shorter horizons and under greater sensitivity to short-term
system dynamics. As a result, short-term activation forecasting becomes both
more relevant and potentially more valuable for participants seeking to adapt
their bidding strategies to real-time market participation. 


Most of the literature on balancing and activation forecasting focuses on direct
prediction of activation volumes and prices
\autocite{backePredictionsPricesVolumes2023,azarangMachineLearningBasedPrediction,plakasPredictionImbalancePrices2025,bankeforsDayaheadModellingElectricity,singhNovelDynamicImbalance2025}.
Activation volumes are relevant to both TSOs and market participants because
they quantify the system's balancing need and the available market
opportunities. Prices are also important to both groups: TSOs care about the
cost of balancing and market efficiency, while participants care because prices
determine settlement outcomes and profitability.

For many demand-side aggregators of flexible distributed resources (e.g., EVs,
heat pumps, batteries), however, the most critical short-term question is often
\emph{not} the exact price level or total regulation volume, but whether the
system will be in a up-regulation, down-regulation, or no-activation state.
Since the marginal costs for demand-side flexible aggregators are small, and
thus they can bid low into the mFRR EAM, they are likely to be activated
whenever there is a need for regulation in their direction. In this sense,
\emph{activation direction} becomes the primary decision variable for flexible
demand-side aggregators, while regulation prices remain important, but secondary
variables that refine expected revenues. The importance of this perspective
increases as aggregators play a larger role in providing system flexibility
under higher renewable penetration
\autocite{caiLowInertiaPower2023,soaresRoleAggregatorsEnergy2025}.



Much of the existing work on mFRR activation volumes take the TSO perspective
\autocite{singhNovelDynamicImbalance2025,
azarangMachineLearningBasedPrediction}. Due to the difference in data access and
decision contexts between TSOs and market participants, one must be cautious
when applying TSO-focused findings to participant-relevant processes. Market
participants are constrained to using only information available at gate closure,
which may exclude certain real-time system signals accessible to TSOs. This
availability concern motivates a participant-centric approach to activation
direction forecasting, ensuring that developed models and insights are directly
applicable to the decision-making contexts of mFRR EAM market participants. 


Activation direction foresight is, in addition to its direct relevance for
demand-side aggregators, also a useful intermediate step towards full volume and
price forecasting. Activation volume and price targets are noisy and difficult
to predict directly. Additionally, because mFRR activation volumes are
zero-inflated, regression-based models tend to struggle as it must learn to
predict both the occurrence and the magnitude of activations simultaneously. An
important insight arises from this: \textbf{activation volumes and prices are
conditional on activation direction.} If the direction of activation is known in
advance, volume and price forecasting can be simplified to a conditional
regression problem, where activation volumes and prices are only predicted given
that an activation in a specific direction is likely to occur. The
zero-inflation problem is thus avoided, as the activation direction model
effectively acts as a filter that separates zero from non-zero activations.



\subsection{Research Questions}

This project is motivated by the need for improved short-term activation
direction foresight among market participants, particularly demand-side
aggregators of flexible resources. The transition to an automated 15-minute mFRR
market creates both the need and the opportunity for enhanced short-term
forecasting capabilities. By focusing on activation direction prediction, this
study aims to provide insights that are directly relevant for select market
participants, and lays the groundwork for more accurate volume and
price forecasting in future work. While these are specific motivations, the
methodological insights gained from this study may also have broader applicability
to other market participant-relevant processes. The likelehoods of up-regulation,
down-regulation, and no activation can, for instance, be useful inputs for 
grander bidding strategy frameworks that seek to optimize participant revenues
in various reserve and balancing markets under uncertainty. Reinforced 
learning-based bidding strategies could potentially leverage activation direction
forecasts as part of their state representation to improve decision-making.
The research questions for this study are:

\textbf{RQ1:} To what extent can mFRR energy activation direction
(up-regulation, down-regulation, or no activation) be predicted at a
15-minute resolution using only information available to market participants
at gate closure under the Nordic automated mFRR market design?
    
\textbf{RQ2:} Which categories of participant-feasible features—particularly
temporal persistence, prices, cross-zonal flows, and production and load
forecasts—contribute most to the short-term predictability of mFRR
activation direction?

\textbf{RQ3:} Does explicit forecasting of mFRR activation direction provide
meaningful value over simple persistence-based heuristics, and
can it serve as a robust intermediate step toward conditional forecasting of
activation volumes and prices?
    






\subsection{Outline}

This report is structured as follows: Chapter 2 provides an overview of the
Nordic balancing markets and machine learning concepts. Chapter 3 reviews
existing literature on imbalance and activation forecasting and identifies the
research gaps addressed in this study. Chapter 4 details the project's
methodology, including data preprocessing, feature engineering, and model
evaluation and selection. Chapter 5 presents the results, while Chapter 6
discusses implications, limitations, and directions for future work. Finally,
Chapter 7 provides concluding remarks summarizing the key findings of the study.