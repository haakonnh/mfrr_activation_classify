\section{Introduction}
\subsection{Background}
The increasing share of weather-dependent renewable energy sources has reduced
the inertia of modern power systems and made short-term system imbalances more 
frequent and difficult to predict \autocite{caiLowInertiaPower2023}. To maintain
frequency quality, Nordic transmision system operators (TSO) rely on various
balancing resources, among which manual Frequency Restoration Reserves (mFRR) play
a key role. The mFRR energy activation market (mFRR EAM) enables market participants
to offer up- and down-regulation reserves that the TSO can activate when sustained
imbalances occur \autocite{entso-eNoricBalancingPhilosophy2024}.

Historically, mFRR activations were manual and scheduled on an hourly basis. On
March 4th 2025, the Nordic region introduced a new automatic mFRR activation
platform with a 15-minute resolution \autocite{ConfirmationMFRREAM2025}. This
reform aligns the Nordic system with European balancing-market standards and
allows activation signals to more closely follow real-time system conditions.
Automated activation of mFRR reserves enables faster response times to
imbalances \autocite{NordicPowerMarket}. This change also means that market 
participants must now make bidding decisions at a finer temporal granularity,
magnifying the importance of quality short-term forecasting.


\begin{figure}[H]
    \includegraphics[width=0.8\textwidth]{Images/dnv_electricity_supply_by_source.png}
    \centering
    \caption{Norway electricity supply by power station and net imports with 
    projections to 2050
    \autocite{EnergyTransitionOutlook}.}
    \label{fig:dnv-electricity-supply-by-source}
\end{figure}


\subsection{Motivation}
mFRR activations have direct economic implications for market participants.
Capacity markets provide stable income, whereas energy activation markets offer
higher but uncertain returns. Anticipating whether mFRR up- or down-regulation
is likely therefore affects how flexible resources should be allocated between
capacity commitments and activation market participation.

This trade-off is particularly relevant for aggregators managing distributed
flexible assets such as electric vehicles, heat pumps, and batteries. Committing
flexibility to the mFRR capacity market secures a fixed payment but binds
resources for the contracted period. In contrast, participation in the
activation market preserves operational flexibility, as resources are
only constrained during actual activations. Aggregators must therefore balance
guaranteed capacity revenues against the expected value of future activation
opportunities, under substantial uncertainty.

The transition to a 15-minute market time unit represents not only a structural
change in market operation but also an opportunity for more informative
short-term decision support. Finer temporal resolution allows activation signals
to better reflect rapidly evolving system conditions, creating richer and more
frequent market signals. At the same time, this increased granularity elevates
the importance of short-term forecasts, as market participants must make
decisions over shorter horizons and under greater sensitivity to short-term
system dynamics. As a result, short-term activation forecasting becomes both
more relevant and potentially more valuable for participants seeking to adapt
their bidding strategies to real-time market participation. 


The majority of literature on balancing activation forecasting focuses on
direct volume and price prediction
\autocite{backePredictionsPricesVolumes2023},
\autocite{azarangMachineLearningBasedPrediction},
\autocite{plakasPredictionImbalancePrices2025},
\autocite{bankeforsDayaheadModellingElectricity},
\autocite{singhNovelDynamicImbalance2025}. Volumes are particularly relevant for
system operators, while imbalance prices are of interest for select market
participants. However, for many participants, especially demand-side aggregators
of flexible distributed resources with near-zero marginal costs, the direction
of activation (up-regulation, down-regulation, or no activation) is the most
critical market signal. These participants do not care about volumes nor prices,
because so long as regulation happens in the expected direction, they will be
activated with high probability due to their low bid prices. The importance of 
this use-case is amplified as the need for demand-side aggregators continues to
grow in response to increasing renewable penetration and system flexibility needs
\autocite{caiLowInertiaPower2023}, \autocite{soaresRoleAggregatorsEnergy2025}.


Much of the existing work on mFRR activation volumes take the TSO perspective
\autocite{singhNovelDynamicImbalance2025},
\autocite{azarangMachineLearningBasedPrediction}. Due to the difference in data
access and decision contexts between TSOs and market participants, one must be
cautious when applying TSO-focused findings to participant-relevant processes.
Market participants are constrained to using only information available at bid
close, which may exclude certain real-time system signals accessible to TSOs.
This availability concern motivates a participant-centric approach to activation
direction forecasting, ensuring that developed models and insights are directly
applicable to the decision-making contexts of mFRR EAM market participants. 


Activation direction foresight is, in addition to its direct relevance for
demand-side aggregators, also a useful intermediate step towards full volume and
price forecasting. Activation volume and price targets are noisy and difficult
to predict directly. Additionally, because mFRR activation volumes are
zero-inflated, regression-based models tend to struggle as it must learn to
predict both the occurrence and the magnitude of activations simultaneously. An
important insight arises from this: \textbf{activation volumes and prices are
conditional on activation direction.} If the direction of activation is known in
advance, volume and price forecasting can be simplified to a conditional
regression problem, where only non-zero activations need to be predicted. 
The zero-inflation problem is thus avoided, as the activation direction 
model effectively acts as a filter that separates zero from non-zero
activations. Regression-based models can then focus solely on predicting
activation price and volume magnitudes, given that an activation in a specific
direction is likely to occur. 


This project is motivated by the need for improved short-term activation
direction foresight among market participants, particularly demand-side
aggregators of flexible resources. The transition to an automated 15-minute mFRR
market creates both the need and the opportunity for enhanced short-term
forecasting capabilities. By focusing on activation direction prediction, this
study aims to provide insights that are directly relevant for select market
participants, and lays the groundwork for more accurate volume and
price forecasting in future work. While these are specific motivations, the
methodological insights gained from this study may also have broader applicability
to other market participant-relevant processes.  

\subsection{Research Statement (or Question?)}
This study investigates the feasibility of short-term forecasting of mFRR energy
activation direction — up-regulation, down-regulation, or no activation — at a
15-minute resolution using only information available to market participants
under the Nordic automated 15-minute mFRR energy activation market design. The
focus is on a participant-centric perspective, reflecting the information
availability constraints and decision contexts faced by aggregators of flexible
distributed resources in the mFRR EAM. By treating activation direction as the primary
forecasting target, the study aims to assess whether meaningful predictive
signals exist at short time horizons following the transition to automated mFRR
activation, and whether such forecasts can serve as practical decision support
for market participants. In addition, the study evaluates the role of activation
direction forecasting as an intermediate step toward more accurate conditional
forecasting of activation volumes and prices in future work.

\subsection{Outline}

This report is structured as follows: Chapter 2 provides an overview of the
Nordic balancing markets and machine learning concepts. Chapter 3 reviews
existing literature on imbalance and activation forecasting and identifies the
research gaps addressed in this study. Chapter 4 details the project's
methodology, including data preprocessing, feature engineering, and model
evaluation and selection. Chapter 5 presents the results, while Chapter 6
discusses implications, limitations, and directions for future work. Finally, 
Chapter 7 concludes the report. 