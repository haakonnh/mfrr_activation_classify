\section{Introduction}
\subsection{Background}
The increasing share of weather-dependent renewable energy sources has reduced
the inertia of modern power systems and made short-term system imbalances more 
frequent and difficult to predict \autocite{caiLowInertiaPower2023}. To maintain
frequency quality, Nordic transmision system operators (TSO) rely on on various
balancing resources, among which manual Frequency Restoration Reserves (mFRR) play
a key role. The mFRR energy activation market (mFRR EAM) enables market participants
to offer up- and down-regulation reserves that the TSO can activate when sustained
imbalances occur \autocite{entso-eNoricBalancingPhilosophy2024}.

Historically, mFRR activations were manual and scheduled on an hourly basis.
On 4 March 2025, the Nordic region introduced a new automatic mFRR activation
platform with a 15-minute resolution \autocite{ConfirmationMFRREAM2025}. This reform
aligns the Nordic system with European balancing-market standards and allows
activation signals to more closely follow real-time system conditions. Automated
activation of mFRR reserves enables faster response times, making it possible 
to adress quarter-hourly imbalances effectively \autocite{NordicPowerMarket}.



\begin{figure}[H]
    \includegraphics[width=0.8\textwidth]{Images/dnv_electricity_supply_by_source.png}
    \centering
    \caption{Norway electricity supply by power station and net imports with 
    projections to 2050 \
    \autocite{EnergyTransitionOutlook}}
    \label{fig:dnv-electricity-supply-by-source}
\end{figure}

\textbf{Struggling with what to include in background as I feel like much of interesting
background information kind of fits in motivation.}


\subsection{Motivation}
mFRR activations have direct economic implications for market participants.
Capacity markets provide stable income, whereas energy activation markets offer
higher but uncertain returns. Anticipating whether mFRR up- or down-regulation
is likely therefore affects how flexible resources are allocated between
capacity commitments and activation market participation.

This trade-off is particularly relevant for aggregators managing distributed
flexible assets such as electric vehicles, heat pumps, and batteries. Committing
flexibility to the mFRR capacity market secures a fixed payment but binds
resources for the contracted period. In contrast, participation in the
activation market preserves greater operational flexibility, as resources are
only constrained during actual activations. Aggregators must therefore balance
guaranteed capacity revenues against the expected value of future activation
opportunities, under substantial uncertainty.

The transition to a 15-minute market time unit represents not only a structural
change in market operation but also an opportunity for more informative
short-term decision support. Finer temporal resolution allows activation signals
to better reflect rapidly evolving system conditions, creating richer and more
frequent market signals. At the same time, this increased granularity elevates
the importance of forecasts, as market participants must make decisions over
shorter horizons and under greater sensitivity to short-term system dynamics. As
a result, short-term activation forecasting becomes both more relevant and
potentially more valuable for participants seeking to adapt their bidding
strategies to real-time market participation.

\textbf{Reinforcement learning motivation here?}



\subsection{Research Question}

The central research question of this study is: To what extent can short-term
mFRR activation direction (up-regulation, down-regulation, or no activation) be
predicted at 15-minute resolution using only information available to market
participants at decision time?

\subsection{Outline}

This report is structured as follows: Chapter 2 provides an overview of the
Nordic balancing markets and machine learning concepts. Chapter 3 reviews
existing literature on imbalance and activation forecasting and identifies the
research gaps addressed in this study. Chapter 4 details the project's
methodology, including data preprocessing, feature engineering, and model
evaluation and selection. Chapter 5 presents the results, while Chapter 6
discusses implications, limitations, and directions for future work.