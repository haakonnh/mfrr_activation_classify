\section{Introduction}
\subsection{Background}
Electric power systems increasingly rely on fast-acting flexible resources to maintain
frequency quality as renewable energy sources displace traditional synchronous generation 
\autocite{caiLowInertiaPower2023} (Figure \ref{fig:dnv-electricity-supply-by-source}).
In the Nordic power markets, manual Frequency Restoration Reserves (mFRR) aplay a central
role in restoring the system balance after unforeseen production and demand imbalances. 
The mFRR energy activation market determines when upward or downward balancing energy is
activated, and these activations can significantly influence market participants' 
revenues. 


\begin{figure}[H]
    \includegraphics[width=0.8\textwidth]{Images/dnv_electricity_supply_by_source.png}
    \centering
    \caption{Norway electricity supply by power station and net imports with 
    projections to 2050 \
    \autocite{EnergyTransitionOutlook}}
    \label{fig:dnv-electricity-supply-by-source}
\end{figure}

The mFRR energy activation market has historically been manual and scheduled on an hourly 
basis. 
On March 4th 2025, the Nordic automatic mFRR activation market with quarter-hourly 
resolution was introduced \autocite{ConfirmationMFRREAM2025}. The shift to 15-minute 
intervals aims to align the Nordic system with European balancing-market standards and 
to allow activation signals to follow system conditions and imbalances more closely.


\subsection{Motivation}
mFRR activations are irregular events with significant economic implications, thus
accurately predicting their occurrence would provide substantial benefits for market
participants. For aggregators
of distributed flexible resources, such as home batteries and electric vehicles,
knowing when mFRR activations are likely to occur can inform bidding strategies in
the electricity balancing markets. Capacity markets and activation markets exist side by side,
each with distinct revenue opportunities. Capacity markets reward participants for being available
to provide reserves, while activation markets compensate for actual reserve activations. 
Accepted bids in capacity markets guarantees revenue regardless of whether an activation
occurs, while activation market bids only yield revenue upon actual activations. 
Bidding strategies involving activation markets are inherently riskier, as revenue 
realization is uncertain. 


    

If an aggregator is able to anticipate when mFRR
activations are unlikely, the aggregator can confidently bid their resources into
the capacity markets without the fear of missing out on high-value activation 
opportunities. Conversely, if an mFRR activation is predicted to be likely, the
aggregator might prioritize bidding into the activation market to capitalize on
potential revenues. Thus, a reliable prediction model for mFRR activations can enhance
market participation strategies and optimize revenue streams for aggregators of
flexible resources. 



The recent reform of the mFRR market in March 2025, transitioning to a Nordic-wide
mFRR activation market with quarter-hourly resolution 
\autocite{ConfirmationMFRREAM2025}, further motivates the
development of a data-driven prediction model. The increased granularity provides
more abundant data points for model training and evaluation, potentially improving
prediction quality. Additionally, the reform introduces more nuanced price signals,
which in theory could be captured by a well-designed prediction model. However,
as the transition is recent at the time of writing, there is limited historical data
available at the new resolution. This scarcity of data presents a challenge for
model development, as machine learning models typically require substantial amounts
of data to learn effectively.
\textit{bid acceptance not equal to activation}
\textbf{not only looking into prices and volumes.}



The aim of this specialization project is, thereby, to provide further work with
a mFRR activation prediction model that can be applied to comprehensive and
realistic market strategies for aggregators of distributed flexible resources.
One possability is to apply reinforcement learning techniques to develop 
near-optimal bidding strategies based on the predictions from the model. This is
an interesting avenue for future research as it, if successful, could provide
significant economic benefits for aggregators and enhance the overall efficiency
of the electricity balancing markets.

\textbf{Handling uncertainty important for further work, near real time, continuous.
Need to better implement the "near real time" aspect.}
\subsection{Research Question}

The central research question of this study is: How well can mFRR up- and 
down-regulation activations be predicted  at 15-minute resolution using only 
real-time-available system state features and historical activation/price data?
What is the relative contribution of these systemic features compared to lag-based
features?

\subsection{Outline}

This report is structured as follows: Chapter 2 provides an overview 
of the Nordic balancing markets and related work. Chapter 3 reviews existing literature 
on imbalance and activation forecasting and identifies the research gaps addressed in 
this study. Chapter 4 details the modelling methodology, including data preprocessing,
feature engineering, and model evaluation and selection. 
Chapter 5 presents the results and performance 
evaluation, while Chapter 6 discusses implications, limitations, and directions for 
future work.