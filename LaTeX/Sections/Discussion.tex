\newpage
\section{Discussion}

\subsection{Summary of Findings}

This study demonstrates that short-term mFRR activation direction at a 15-minute
resolution is predictable to a limited but meaningful extent using only
participant-feasible information available at gate closure. Across all evaluated
models, short-term predictability is dominated by temporal persistence in recent
activation states, with a simple persistence-based baseline proving difficult to
outperform in macro-averaged classification performance.

Clear differences in predictability are observed across activation classes.
Down-activations are the most consistently predictable class: on the test set,
69\% of all down-activations, which constitute 37.2\% of all activations, are
correctly identified by the best-performing model, with 62\% precision
(Table \ref{fig:catboost-confusion-val-test}). Up-activations are comparatively
less predictable, reflecting their more sporadic and less persistent nature.

Other at-gate-closure features, including prices, cross-zonal flows, and production
and load forecasts, provide only marginal incremental value once persistence is
accounted for. Overall, the results indicate that activation direction forecasting
under strict information constraints is most effective when treated as a
short-horizon, persistence-aware classification problem, and that class-conditional
performance differences are critical for assessing practical usefulness.

Finally, when comparing economic relevance via the price difference analysis, the
machine learning model captures more profitable activations than the naive model
among its correct predictions. On the test set, Table \ref{tab:price-spread-stats-tail20}
compared with Table \ref{tab:price-spread-stats} shows higher mean price spreads for
both up-activations (17.98 EUR/MWh vs 15.83 EUR/MWh) and down-activations (-12.34
EUR/MWh vs -11.34 EUR/MWh), suggesting that the model identifies more lucrative
opportunities beyond what is explained by simple persistence patterns.


\subsection{Answers to Research Questions}

\textbf{RQ1: To what extent can mFRR energy activation direction be predicted at a 15-minute 
resolution using only information available to market participants at gate closure?}

The results indicate that mFRR activation direction is predictable to a limited
but non-negligible extent at gate closure when evaluated at a 15-minute resolution.
Across all evaluated models, short-term predictability is primarily driven by
temporal persistence in recent activation states. Models are able to correctly
classify a substantial share of down-activation events, with the best-performing
model correctly predicting 69\% of all down-activations in the test set with
62\% precision. Up-activations are more difficult to predict, with lower
recall and precision. 


\textbf{RQ2: Which categories of participant-feasible features contribute most to the 
short-term predictability of mFRR activation direction?}

Feature importance analyses show that short-term temporal persistence features
dominate the predictive signal. Recent activation states consistently emerge as
the most influential predictors across the models, reflecting strong
auto-correlation in mFRR activation behavior. Other feature categories,
including prices, cross-zonal flows, and production and load forecasts, provide
only marginal incremental information once persistence is accounted for. Feature
correlation analyses further suggest that many of these additional features
correlate with recent activation states, limiting their unique predictive value.
PCA results corroborate these findings, indicating that most variance in the
feature set can be captured by a significantly reduced number of components.

\textbf{RQ3: Does explicit forecasting of mFRR activation direction provide 
value over simple persistence-based baselines, and can it serve as an intermediate step 
toward conditional forecasting of volumes and prices?}

Explicit activation direction forecasting provides limited but potentially
meaningful incremental value over a pure persistence baseline, particularly in
class-specific performance and decision-relevant contexts. While overall macro
performance gains are modest, the models demonstrate improved discrimination for
certain activation states, most notably down-activations, relative to the naive
baseline. Combined with the observed price differences conditional on correctly
predicted activation direction, this suggests that direction forecasting can
support participant decision-making in settings where correct directional
signals entail economic value. Furthermore, treating activation direction as a
separate classification problem provides a conceptually and methodologically
useful intermediate layer for future conditional forecasting of activation
volumes and prices, where direction uncertainty otherwise complicates direct
regression approaches.

\subsection{Implications}

A central implication of this study is that \textbf{mFRR down-activation direction
exhibits relatively strong short-term predictability under participant-feasible
information constraints in NO1}. The evaluated models achieve materially higher
recall and precision for down-activations than for up-activations, indicating that
down-regulation events tend to persist over short horizons and are therefore more
amenable to anticipation at bid close.

These findings are particularly relevant for demand-side aggregators of flexible
resources, for whom short-term activation direction constitutes a key operational
signal. For such participants, the results suggest that forecast-driven decision
support is most reliable when used to \textbf{prioritize down-regulation-oriented
mFRR strategies}. Directional forecasts can serve as a filtering mechanism that
helps identify intervals with elevated likelihood of down-activation, supporting
decisions on whether flexibility should be reserved for mFRR participation or
allocated to alternative uses.

While correct up-activation predictions are associated with larger deviations from
day-ahead prices and therefore potentially higher value per event, the lower
predictability of up-activations makes such opportunities more difficult to
exploit consistently under gate-closure information constraints. As a result,
down-regulation strategies are likely to offer more robust and repeatable decision
support for demand-side aggregators in practice.

More broadly, the strong role of persistence implies that simple heuristics remain
competitive, and any activation forecasting model should be benchmarked against a
persistence-based baseline. The limited incremental value of other at-gate-closure
features further suggests that performance improvements are more likely to arise
from richer temporal modelling of activation regimes or access to better-aligned
information closer to real time, rather than from incremental expansion of coarse
gate-closure feature sets.




\subsection{Limitations}
Several factors can explain why performance gains over a persistence baseline are
limited: (i) class imbalance; (ii) mFRR activations exhibit strong
auto-correlation; (iii) information constraints at gate closure, which remove the
most informative recent observations; (iv) varying data availability and
granularity; and (v) limited data span and representativeness.

Class imbalance was addressed through decision threshold tuning, but more
sophisticated rebalancing techniques could have been explored. It seems,
however, that the ``up'' class is inherently more difficult to predict due to it
being more sporadic and less persistent than the other classes.

The strong auto-correlation in activation patterns, especially for down- and
no-activations, means that recent activation states are highly predictive of
future states. Other more nuanced features may provide only marginal, and
potentially noisy, signals that a model may not value highly. Persistence
features may therefore have overshadowed other potentially useful features.
Attempts were made to exclude persistence features to force models to learn from
other signals, but this led to substantially worse performance. This suggests
that other features do not contain sufficient predictive information beyond what
is already captured by persistence.

The information constraints at gate closure are a significant limitation. This
study deliberately restricts inputs to publicly available information, while
actual market participants may have access to additional and more recent
proprietary signals, which could improve short-horizon predictability beyond
what is achievable here. Activation directions are predicted an hour in advance,
thus no real-time signals are guaranteed to represent system conditions at the
time of activation. This creates an information gap that models cannot bridge.
Recent activations are the largest exceptions as they often persist across the
gap. In addition, data publication delays can change over time: during early
stages of this project, Nord Pool data appeared to be released with a delay of
four 15-minute time slots, whereas it now seems to be published with a delay of
two time slots. Such changes affect what is realistically
``participant-feasible'' at gate closure and can influence both model
performance and the comparability of results across periods. Varying data
granularity also poses challenges, as some features are only available at hourly
resolution and must be resampled. This removes important intra-hour dynamics
that could aid prediction. Some data streams have the additional limitation of
being released at specific times during the day, limiting availability dependent
on time of day (see Table \ref{tab:data_availability}).

It is difficult to know whether or not the dataset used in this study is
representative enough to capture a sufficient variety of patterns. The final
sample period covers 10 months at 15-minute resolution, amounting to $28\,800$
data points. This is not a small dataset per se, but as the dataset consists of
under a year of data, seasonal patterns may not be fully captured. Features that
indicate seasonality or patterns that may repeat annually are not possible to
learn from this dataset. Datasets on such short timeframes also risk being
poorly generalizable to future periods. A similarly featured dataset trained
over multiple years may yield better results. It is also important to note that
this study only considers the price area NO1. Other areas may have different
characteristics that could affect predictability, such as less imbalance, less
persistence, or differing system dynamics.

\subsection{Future Work}
Section \ref{sec:classification-threshold-adjustment-results} showed that
adjusting classification thresholds post-training can improve minority-class and
macro performance. Future work could explore other class imbalance techniques,
such as \textbf{integrating cost-sensitive learning} directly into model training to
better handle class imbalance \autocite{2CostsensitiveLearning}. There is also
room for improved upsampling or downsampling techniques to rebalance the
training data.

Although this study incorporated a diverse set of features available at gate
closure and machine learning-based models, future research could explore: (a) more
nuanced weather-related features to
better capture wind power production deviations and intermittency effects; (b) a
more thorough investigation of the predictability of non-persistence-driven
features for mFRR activations; and (c) alternative model architectures, such as
recurrent neural networks or transformers, that may better capture temporal
dependencies.

This project focused on predicting discrete mFRR activation direction at a
15-minute resolution. Future work could extend this to a \textbf{probabilistic
classification framework}, providing likelihood estimates for each activation
direction. Such frameworks are perhaps more difficult to evaluate, but could
provide more useful information for market participants. Additionally, more
comprehensive bidding strategy architectures, such as reinforcement
learning-based agents, could utilize probabilistic distributions to inform
optimal decisions under uncertainty.

