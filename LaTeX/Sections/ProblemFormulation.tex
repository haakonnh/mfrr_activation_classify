\newpage
\section{Problem Formulation}\label{sec:problem-formulation}
	\textit{Note: This section is a working draft and may be removed later; 
    it currently preserves some ideas for potential reuse.}
The primary objective of this study is to develop a machine learning model
capable of predicting manual Frequency Restoration Reserves (mFRR) activations
in the Nordic electricity market, specifically for the bidding zone NO1 in 
Norway. The problem is formulated as a classification task, where the model 
aims to classify time intervals into distinct categories based on the occurrence
of mFRR activations. The classification task is approached in two stages: 
binary classification and ternary classification. 

\subsection{Binary Classification Problem}
In the binary classification problem, the model predicts whether an mFRR
upregulating activation occurs within a given time interval. This is a
straightforward classification task with two possible outcomes: activation 
or no activation. This formulation is a good starting point, as it allows 
for exploration of the model's ability to identify patterns associated with
upregulating activations specifically. Binary models are also generally easier
to train and tune, as they focus on a single class of interest, thus 
providing a base understanding of expected model performance before moving on
to the more complex ternary classification problem.

\subsection{Ternary Classification Problem}
In the ternary classification problem, the model extends its predictive
capabilities to include both upregulating and downregulating mFRR activations.
This introduces a third class to the classification task, resulting in three
possible outcomes: upregulating activation, downregulating activation, or no
activation. This formulation captures the full spectrum of mFRR activation
dynamics, making it ideal for practical applications in the electricity market.
This approach, however, is more complex than the binary classification problem,
as the binary models can be more focused on the specific characteristics of
upregulating activations. The ternary models, on the other hand, must learn to
distinguish between three classes, which can be more challenging, possibly
causing the model to be less certain in its predictions.

