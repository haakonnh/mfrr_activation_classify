\newpage
litterature review,
compare litterature,
gap,

\section{Problem Formulation}\label{sec:problem-formulation}
	\textit{Note: This section is a working draft and may be removed later; 
    it currently preserves some ideas for potential reuse.}
The primary objective of this study is to develop a machine learning model
capable of predicting manual Frequency Restoration Reserves (mFRR) activations
in the Nordic electricity market, specifically for the bidding zone NO1 in 
Norway. The problem is formulated as a classification task, where the model 
aims to classify time intervals into distinct categories based on the occurrence
of mFRR activations. The classification task is approached in two stages: 
binary classification and ternary classification. 

\subsection{Binary Classification Problem}
In the binary classification problem, the model predicts whether an mFRR
upregulating activation occurs within a given time interval. This is a
straightforward classification task with two possible outcomes: activation 
or no activation. This formulation is a good starting point, as it allows 
for exploration of the model's ability to identify patterns associated with
upregulating activations specifically. Binary models are also generally easier
to train and tune, as they focus on a single class of interest, thus 
providing a base understanding of expected model performance before moving on
to the more complex ternary classification problem.

\subsection{Ternary Classification Problem}
In the ternary classification problem, the model extends its predictive
capabilities to include both upregulating and downregulating mFRR activations.
This introduces a third class to the classification task, resulting in three
possible outcomes: upregulating activation, downregulating activation, or no
activation. This formulation captures the full spectrum of mFRR activation
dynamics, making it ideal for practical applications in the electricity market.
This approach, however, is more complex than the binary classification problem,
as the binary models can be more focused on the specific characteristics of
upregulating activations. The ternary models, on the other hand, must learn to
distinguish between three classes, which can be more challenging, possibly
causing the model to be less certain in its predictions.

\subsection{Restrictions}\label{sec:restrictions}
In an ideal world, a market participant would be able to choose to bid or
not in real time. The mFRR activation market, however, is not ideal in
this sense. Bids are accepted for time slots $t+4$ intervals into the
future, i.e. one hour into the future. This greatly restricts market
actors in terms of how fast they can act on recent information. In fact,
at the time of bid closing for a time slot $t+4$, only mFRR activation
data from $t-3$ and earlier is available. Thus there are seven 15-minute
intervals of unavailable data. This data would be the most
useful for the model, since activation data at time $t$ correlates with
the system state and activations close to $t$.

It would, for instance, be considerably easier to identify streaks of
activations, as activations often occur in long contiguous sequences. In the real
world scenario, data from $t-3$ and before is still useful, but can often be
misleading and cannot be trusted in isolation. The dataset developed in
this project investigates ways to provide the model with crucial context
about the system's state of stress. The intuition is that data such as
physical cross-zonal flows in and out of NO1 may correlate with NO1 mFRR
activations. When many such features are combined, the model may develop
complex relationships between them, enhancing its predictive performance.

Most of the available and useful data are not available in real time. For
instance, consumption and production data are often published with a delay of
several hours. The model must therefore rely on features that are available
in real time, or with minimal delay. Such features often take the form of
forecasts, which are available ahead of time. Forecasts are inherently
uncertain, but they still provide valuable information about the expected
system state.

\subsection{Class Imbalance}\label{sec:class_imbalance}

