\newpage
\subsection{Feature Engineering}
Features are attributes in a dataset that describe each data point. A dataset
for predicting a person's income could, for instance, have features like gender,
job type, and age. Then, each data point represents a person and relevant
information about the person in terms of the a target variable, e.g., income .
In this project, each data point represents a specific time stamp in the mFRR
activation dataset, and the features describe the system state at that time.
This includes information such as electricity demand, generation capacity, and
market prices, all of which can influence mFRR activations.

Already available features can be transformed to create new higher-level
features that may better capture the underlying patterns in the data. For
example, if one has features for year of death and year of birth, a new feature
for age at death can be created by subtracting the year of birth from the year
of death. This is known as feature construction [37]. Features can be
constructed in various ways, such as through mathematical operations or
aggregations. This project leverages this concept to create new features that
may enhance the model's predictive capabilities.

Feature selection is crucial. There are many features that may seem useful and
relevant in isolation, but sometimes they mislead the models, or they work
poorly in combination with other seemingly good features. Theoretical analysis
of the usefulness of certain features can be helpful, but only trial-and-error
together with feature-importance analysis will uncover the features' actual
impact on performance. 

\subsubsection{Time Restrictions and Data Availability}

Short-term activation prediction is constrained by the limited availability of
recent system data at prediction time. mFRR EAM bidding for a specific MTU
closes 45 minutes prior, so at time $t$ we can only act on predictions for the
interval $t+4$ and later. This restriction causes a great amount of uncertainty.
Even the best approximation of the current system state is several 15-minute
intervals old at the target delievery time.

Many data sources have reporting delays, meaning that the most recent data
points are not yet published at prediction time. For example, mFRR EAM
activation data is available with a delay of approximately one hour. This means
that at time $t$, the most recent activation data available is from time $t-4$.
However, since mFRR imbalance volumes are known to be highly auto-correlated
\autocite{backePredictionsPricesVolumes2023}, even delayed activation data can
provide valuable information about current trends. Figure
\ref{fig:time-restrictions} illustrates the time restrictions on feature
availability for predicting mFRR activations at time $t+4$ based on data
available at time $t$. Most real-time-relevant information is delayed by an
hour, while accepted balance market activation volume and price are available at
$t-1$, making it the most recent activation-related data point accessible at
prediction time.

\begin{figure}[H]
    \centering
    \includegraphics[width=0.9\textwidth]{Images/TimeDiagram.drawio.png}
    \caption{Illustration of time restrictions on feature availability for
    predicting mFRR activations at time $t+4$ based on data available at time
    $t$.}
    \label{fig:time-restrictions}
\end{figure}

An additonal complication is that some data sources are made available only on
certain times of the day. mFRR capacity market data for the following day, for
instance, is published after the capacity market gate closure at 23:00 each day.
At 01:00 the next day, capacity market data is available for the entire day
ahead, but at 22:00 the same day, only data for the next two hours is available.
This data is thus not only available for target time $t+4$. This complicates
feature engineering, as the model must be able to handle features that are only
partially available depending on the time of day. Table
\ref{tab:data_availability} summarizes the gate closure times of various market
datasets relative to the delivery day $D$. Market gate closure may not align
perfectly with actual market participant availability and release times, but
they provide a good approximation. In this project, these data release timing
constraints are not taken into account when engineering features, but they
should be considered in future work to ensure that models can operate under
real-world data availability conditions. 

\begin{table}[h]
\centering
\caption{Market gate closures and availability times relative to delivery day $D$
\autocite{IntradayAuctionsSIDC}.}
\label{tab:data_availability}
\begin{tabular}{lll}
\hline
\textbf{Market / Dataset} & \textbf{Gate Closure Time} & \textbf{Relative to Delivery} \\
\hline
mFRR Capacity Market      & 23:00 & $D-1$ \\
Day-Ahead Auction         & 12:45 & $D-1$ \\
Intraday Auction 1 (IDA1) & 15:00 & $D-1$ \\
Intraday Auction 2 (IDA2) & 22:00 & $D-1$ \\
Intraday Auction 3 (IDA3) & 10:00 & $D$   \\
\hline
\end{tabular}
\end{table}

During the course of this project, it seems that many data sources and their 
availabilities have change slightly. Balance market market activation data 
is, for instance, now available for $t-3$ instead of $t-4$. Such changes should
be taken into account in future work, but it is unlikely that they will have
a major impact on the overall findings of this study.



\paragraph{Activation lag features.}
Activation lag features are the most important lag features for this problem, as
they convey important information about recent temporal activation trends. They
are, however, restricted by the real-time limitations, so the model may only use
activation lag features from $t-4$ and earlier for predicting activations at time
$t+4$. As a result, lag features for upregulating and downregulating activations
are created for time steps $t-4$, $t-5$, $t-6$, ..., $t-9$. These features are useful by
themselves, but they also serve as a basis for creating other features that
capture activation trends more effectively.

\paragraph{Persistence (activation streak length).} Lag features indicate what
happened in the most recent intervals, but they do not express whether the
system has been in a sustained activation phase. For the model to understand
such trends, it would need to look at several lagged activation features
simultaneously and infer whether there has been a streak of up- or
down-activations. To capture this behaviour succinctly, a set of persistence
features is included. These measure how long the latest sequence of up-, down-,
or no-activations has lasted, based only on the activation data that are
available at bid time. The idea is straightforward: if down-activations occurred
in several consecutive intervals leading up to and including $t-4$, the
down-persistence value reflects the length of that streak, as visualized in
figure 13. The same applies for up-activations and no-activations. These
persistence features aim to capture the tendency for mFRR activations to occur
in clusters, as periods of system stress often lead to repeated activations. By
condensing this pattern into a single value for each direction, the model is
given a clearer representation of ongoing activation dynamics than lagged
indicators alone can provide.

\begin{figure}[H]
    \centering
    \includegraphics[width=0.9\textwidth]{Images/Persistence.png}
    \caption{ Illustration of down persistence feature calculation based on
lagged activation features. Here, the down persistence at time $t$ is 2, as there
have been down-activations in the two most recent intervals ($t-4$ and $t-5$),
before an interval with no activation at $t-6$.}
    \label{fig:persistence_illustration}
\end{figure}

Persistence feature capture consecutive activation trends effectively, but they
do have limitations. Singular lagged activation features are still useful in the
case of intermittent activations that do not form long streaks. Additionally,
persistence features do not convey the magnitude of recent activations, only
their occurrence. How all mFRR activation-related features are used in
conjunction will be subject to experimentation and analysis during model
development.

\subsubsection{Cross-zonal flow features}
Cross-zonal flow features capture information about electricity flows between
different zones or regions in the power grid- in this case, in and out of the
NO1 bidding zone. These flows can indicate the grid's stress level and influence
mFRR activations. For instance, high inflows into NO1 may signal increased
demand or generation shortages, potentially causing upregulating activations.
Conversely, high outflows may indicate surplus generation, potentially causing
downregulating activations. It is unlikely, though, that cross-zonal flow
features alone can predict activations. Combining them with available transfer
capacity provides a picture of how close the grid is to its operational limits.
For instance, if the inflow into NO1 is close to the maximum available transfer
capacity, only a small margin remains for additional inflows, which could
increase the likelihood of upregulating activations. Such situations often occur
in zones that are short, i.e. zones where consumption exceeds production. In
such cases, the grid operator may need to activate expensive mFRRreserves to
maintain grid stability when no more cheap imports are possible. It is important
that the models developed in this project are able to capture these kind of
relationships as they are among the most valuable for a potential user of the
models.

\paragraph{Capacity-normalized cross-zonal flow.} 

Raw cross-zonal flow magnitudes are not comparable across interconnections or
over time because each line has different capacity and the available transfer
capacity (ATC) varies. The same absolute flow can be insignificant on a
high-capacity interconnection but critical on a constrained one. There is thus
value in transforming such features to be on a similar scale
\autocite{NumericalDataNormalization}. Flows are expressed as a ratio to
the relevant directional ATC.
Let $F_i(t)$ be the flow on interconnection $i$ at time $t$, taken as positive when power flows into
NO1 and negative when it flows out. Let ATC$_i(t)$ be the available transfer capacity for that
interconnection at time $t$. The capacity-normalized flow is then
\begin{equation*}
F^i_{\text{ratio}}(t) = \dfrac{F_i(t)}{\mathrm{ATC}_i(t)}
\end{equation*}

Values of $F^i_{\text{ratio}}(t)$ close to 1 mean that the inflow is close to
the capacity, values close to -1 mean that the outflow is close to the capacity,
and values near 0 mean that the net flow is small compared to the available
capacity.

\subsubsection{Temporal features}
Temporal features capture time-related patterns in the data. These features help
the model un derstand how mFRR activations vary with time, such as daily or
weekly cycles. Basic temporal features include hour of the day, day of the week,
and month of the year. These features allow the model to learn patterns related
to specific times. mFRR activations could, for instance, be caused by completely
different factors during peak hours on weekdays compared to off-peak hours on
weekends. Temporal features like these are most often represented using cyclical
encoding to reflect their periodic nature. For example, 1 AM and 11 PM are close
in time, even though their numerical representations (1 and 23) are far apart.
Cyclical encoding uses sine and cosine transformations to capture this
periodicity \autocite{pelletierCyclicalEncodingAlternative2024}. Hourly features
are, for instance, encoded as:

\begin{equation*}
\begin{aligned}
\mathrm{Hour}_{\sin} &= \sin\left(2\pi \cdot \frac{\mathrm{Hour}}{24}\right), \\
\mathrm{Hour}_{\cos} &= \cos\left(2\pi \cdot \frac{\mathrm{Hour}}{24}\right).
\end{aligned}
\end{equation*}


Monthly features are encoded similarly, using 12 as the divisor instead of 24.





\subsubsection{Price features}
Price features capture information about the various electricity market prices.
The mFRR activation market is closely linked to other electricity markets, such
as the day-ahead market, the intraday market, and the aFRR market. Most prices
may not have direct impacts on activations, but constructing features that
capture relationships between prices, the model may be able to infer
system stress levels that could lead to mFRR activations. Large discrepencies
between day-ahead prices and intraday prices may, for instance, indicate
unexpected changes in supply or demand, which should correlate with mFRR
activations. Similarly, the difference between aFRR prices and mFRR prices may
provide insights into the relative costs of balancing services, which could
influence activation decisions. Table \ref{tab:price_features_overview} summarizes
the engineered price features and their underlying intuition.


\begin{table}[h]
\centering
\caption{Overview of engineered price features and their intuition.}
\label{tab:price_features_overview}
\begin{tabular}{@{}ll@{}}
\toprule
\textbf{Feature name} & \textbf{Intuition} \\
\midrule
Day-ahead price &
Baseline wholesale price level for the MTU. \\

Intraday prices (IDA1, IDA2, IDA3) &
Near-real-time price; reflects late system updates. \\

aFRR up/down price &
Balancing cost proxy; indicates system stress. \\

Lagged up/down mFRR prices &
Recent mFRR price trends; activation cost signal. \\


Up/down to day-ahead spread and ratio &
Relative balancing cost vs.\ baseline price. \\

DA--ID spread and ratio &
Late deviations vs.\ schedule; forecast error signal. \\

DA--ID symmetric relative spread & 
Magnitude of late price changes; system volatility. \\


\bottomrule
\end{tabular}
\end{table}


\subsubsection{Production features}

Production features capture information about electricity generation,
particularly from renewable sources like wind power. Wind power production
features were considered promising candidates for predicting mFRR activations,
as wind power is intermittent and can cause sudden changes in supply. figures
14a and 14b show values of realized wind production and wind share (wind
production as a fraction of total production) plotted against the distribution
of mFRR activations. These figures indicate that there is no direct correlation
between wind production and mFRR activations. The existence of such a
correlation would have made it easy for the model to leverage wind production
features for predicting activations. The hope is, however, that wind production
features will prove useful when combined with other features, as the model
captures complex relationships between features. \textbf{Should be in 
EDA?}

\begin{figure}[H]
    \centering
    \begin{subfigure}{0.48\textwidth}
        \centering
        \includegraphics[width=\linewidth]{Images/actual_wind_doesnt_impact_prediction.png}
        \caption{Realized wind production plotted against mFRR activation
distribution.}
        \label{fig:realized_wind_mfrr}
    \end{subfigure}
    \hfill
    \begin{subfigure}{0.48\textwidth}
        \centering
        \includegraphics[width=\linewidth]{Images/wind_share_impact_activation.png}
        \caption{Forecasted wind production plotted against mFRR activation
distribution.}
        \label{fig:wind_share_mfrr}
    \end{subfigure}
    \caption{Distributions of mFRR activations as functions of realized wind production and wind share.}
    \label{fig:wind_production_distributions}
\end{figure}

\subsubsection{Load features}
Load features capture information about electricity consumption patterns.
Absolute consumption magnitude for NO1 is included as a feature, but there are
many ways to encode consumption in normalized or relative terms. For instance,
consumption can be expressed as a ratio to forecasted consumption, to capture
forecast errors. Consumption can also be expressed as a ratio to relevant
historical consumption values. For the final feature set, a ratio between
current consumption and the average consumption at the same hour throughout the
dataset is used as a feature to capture deviations from typical patterns.

\subsubsection{Interaction features}
Interaction features are created by combining two or more existing features
to capture complex relationships. Many such interactions were experimented with,
mostly by subtracting, multiplying, or dividing pairs of features that were
theoretically expected to have meaningful interactions. Price features were, 
for instance, combined by calculating price spreads or ratios between different 
market prices (day-ahead, intraday, aFRR, mFRR). Consumption and production
features were also combined to create features that capture net load or
supply-demand imbalances. Many more are possible, but only the most promising 
interaction features were included in the final feature set to make feature 
analysis more interpretable.
