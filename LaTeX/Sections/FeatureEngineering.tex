\newpage
\subsection{Feature Engineering}
Features are attributes in a dataset that describe each data point. A dataset
for predicting a person's income could, for instance, have features like
gender, job type, and age. Then, each data point represents a person and
relevant information about the person in terms of the target variable
-- income. In this project, each data point represents a specific time stamp
in the mFRR activation dataset, and the features describe the system state at
that time. This includes information such as electricity demand, generation
capacity, and market prices, all of which can influence mFRR activations.

Already available features can be transformed to create new higher-level
features that may better capture the underlying patterns in the data. For
example, if one has features for year of death and year of birth, a new feature
for age at death can be created by subtracting the year of birth from the
year of death. This is known as \textit{feature construction} 
\autocite{liuFeatureSelectionKnowledge2012}. Features can be constructed in 
various ways, such as through mathematical operations or aggregations. This
project leverages this concept to create new features that may enhance the model's
predictive capabilities.

Feature selection is crucial. There are many features that may seem useful
and relevant in isolation, but sometimes they mislead the models, or they
work poorly in combination with other seemingly good features. Theoretical
analysis of the usefulness of certain features can be helpful, but only
trial-and-error together with feature-importance analysis will uncover the
features' actual impact on performance. Feature selection will be subject to
restrictions outlined in Section \ref{sec:restrictions} to ensure that only
real-time available features are used. 

\subsubsection{Lag features}
A lag feature is a feature that represents the value of a variable at a
previous time step. Lag features are commonly used in time series analysis to
capture \textit{temporal} dependencies and trends in the data 
\autocite{brownleeBasicFeatureEngineering2016}. By including lag
features, the model can leverage historical information to make more informed
predictions about future mFRR activations. For instance, if there was an
upregulating activation in the previous time step, it may indicate a higher
likelihood of another upregulating activation in the current time step.

\paragraph{Activation lag features}
Activation lag features are the most important lag features for this problem,
as they convey important information about recent temporal activation trends. 
They are,
however, restricted by the real-time limitations, so the model may only use
activation lag features from $t-3$ and earlier for predicting activations at
time $t+4$. As a result, lag features for upregulating and downregulating
activations are created for time steps $t-3$, $t-4$, $t-5$, ..., $t-9$. These
features are useful by themselves, but they also serve as a basis for creating
other features that capture activation trends more effectively.

\paragraph{Persistency (streak length)}
Let $t$ denote the reference time such that the model predicts activation at
$t+4$. At bid close, only activation data up to and including $t-3$ are
available. Define binary activation indicators $A^{\uparrow}(\cdot)$ and
$A^{\downarrow}(\cdot)$ for up- and down-regulating activations, respectively.
The up- and down-persistency features are the lengths of the most recent
consecutive activation runs ending at $t-3$: \textbf{TODO: Consider only formulating
persistency with words, as the mathematical expression is more complex
than the concept.}
\[
  S^{\uparrow}(t) = \sum_{j=0}^{\infty} \prod_{i=0}^{j} A^{\uparrow}(t-3-i),\qquad
  S^{\downarrow}(t) = \sum_{j=0}^{\infty} \prod_{i=0}^{j} A^{\downarrow}(t-3-i).
\]
Here, $j$ is the (zero-based) window length and $i$ is the offset index within
that window. The product is $1$ only while all the last $j{+}1$ values are $1$,
so when a $0$ is encountered, the product becomes $0$ for all larger $j$. Thus,
the sum counts how many consecutive intervals back from $t-3$ remain all
ones before the first zero. In words, $S^{\uparrow}(t)$ (resp. $S^{\downarrow}(t)$)
counts how many consecutive up- (resp. down-) activations occurred immediately 
before $t-3$. If there was no activation at $t-3$, the corresponding streak length
is $0$.

\paragraph{Design choice: separate vs. signed.} When designing the persistency
features, a choice had to be made between using two separate non-negative integer
features or a single signed feature (e.g., $S^{\uparrow}(t)-S^{\downarrow}(t)$). 
Separate features avoid conflating direction with magnitude and let the
model learn asymmetric effects. This option may be easier for the model to interpret,
as it provides a clearer distinction between up- and down-regulating activations.
The signed variant is more compact, but it may introduce ambiguity in how the model
interprets the values. It is, however, important to consider the possibility of 
persistency overreliance. If the model relies too heavily on persistency features,
it may overlook other important factors influencing activations. This could lead
to suboptimal predictions, especially in scenarios where activation patterns change.
Only having one persistency feature may reduce this risk, as persistency is
encoded in a single column. Both options will be explored and evaluated during
model development, and the pros and cons of each approach will be assessed based
on empirical performance.

\paragraph{Importance.} Persistency features are perhaps the single most important 
features in the dataset. They provide compact and direct information about recent 
activation trends, making it easier for the model to identify activation streaks. As 
activations often come in long contiguous runs, and even though there are real-time
limitations, the persistency features still provide the model with valuable context.
The longer the streak, the more likely it is that the activation trend will continue.
In some cases, the model might be able to predict an activation solely based on
a high persistency value, without needing to consider other features. 

Much of the motivation behind this project is to develop models that can predict activations
based on system state features rather than just relying on past activation trends.
The reality is, however, that persistency features are extremely dominant, and
they may overshadow the contributions of other features. 

\subsubsection{Cross-zonal flow features}
Cross-zonal flow features capture information about electricity flows between
different zones or regions in the power grid - in this case, in and out of
the NO1 bidding zone. These flows can indicate the grid's stress level and
influence mFRR activations. For instance, high inflows into NO1 may signal
increased demand or generation shortages, potentially causing upregulating
activations. Conversely, high outflows may indicate surplus generation,
potentially causing downregulating activations. It is unlikely, though, that
cross-zonal flow features alone can predict activations. Combining them with
available transfer capacity provides a picture of how close the grid is to its
operational limits. For instance, if the inflow into NO1 is close to the maximum
available transfer capacity, only a small margin remains for additional inflows,
which could increase the likelihood of upregulating activations. Such situations
often occur in zones that are short, i.e. zones where consumption exceeds
production. In such cases, the grid operator may need to activate expensive
mFRR reserves to maintain grid stability when no more cheap imports are possible.
It is important that the models developed in this project are able to capture
these kind of relationships as they are among the most valuable for a potential
user of the models.

\paragraph{Capacity-normalized cross-zonal flow.} Raw cross-zonal flow magnitudes are not 
comparable across interconnections or over time because each line has different capacity 
and the available transfer capacity (ATC) varies. The same
absolute flow can be insignificant on a strong interconnection but critical on a 
constrained one. To obtain a dimensionless, capacity-normalized measure of proximity 
to operational limits—and to make features comparable across borders and time—
flows are expressed as a ratio to the relevant directional ATC.

Let $F_i(t)$ be the \emph{signed} flow for interconnection $i$ at time $t$
(positive into NO1, negative out of NO1). Let $ATC_i(t)\ge 0$ denote the
available transfer capacity magnitude used for normalization (e.g., a symmetric
ATC for interconnection $i$ at time $t$). The capacity-normalized ratio is then
\[
  F_{\text{ratio}}^{i}(t) \,=\, \frac{F_i(t)}{ATC_i(t)}.
\] 
Hence
$F_{\text{ratio}}^{i}(t)\in[-1,1]$ when flows are within limits, approaching $+1$
as inflow nears the capacity and $-1$ as outflow nears the capacity.


\subsubsection{Temporal features}
Temporal features capture time-related patterns in the data. These features help the model
understand how mFRR activations vary with time, such as daily or weekly cycles. 
Basic temporal features include hour of the day, day of the week, and month of the year.
These features allow the model to learn patterns related to specific times. mFRR activations
could, for instance, be caused by completely different factors during peak hours on weekdays
compared to off-peak hours on weekends. Temporal features like these are most often
represented using cyclical encoding to reflect their periodic nature. For example,
1 AM and 11 PM are close in time, even though their numerical representations (1 and 23)
are far apart. Cyclical encoding uses sine and cosine transformations to capture this
periodicity \autocite{pelletierCyclicalEncodingAlternative2024}. 
Hourly features are, for instance, encoded as:
\[
\text{Hour}_{\text{sin}} = \sin\left(2\pi \cdot \frac{\text{Hour}}{24}\right),\qquad
\text{Hour}_{\text{cos}} = \cos\left(2\pi \cdot \frac{\text{Hour}}{24}\right).
\]
Monthly features can be encoded similarly, using 12 as the divisor instead of 24.

\subsubsection{Price features}
Price features capture information about the various electricity market prices. 
The mFRR activaiton market is closely linked to other electricity makets, such as the 
day-ahead market, the intraday market, and the aFRR market. Most prices may not have 
direct impacts on activations, but by crafting features that capture important 
relationships between prices, the model may be able to infer system stress levels
that could lead to mFRR activations. Large discrepencies between day-ahead prices
and intraday prices may, for instance, indicate unexpected changes in supply or demand,
which should correlate with mFRR activations. Similarly, the difference between aFRR prices
and mFRR prices may provide insights into the relative costs of balancing services, 
which could influence activation decisions.

\subsubsection{Load features}


\subsubsection{Production features}
Production features capture information about electricity generation, particularly
from renewable sources like wind power. Wind power production features were
considered promising candidates for predicting mFRR activations, as wind power is
intermittent and can cause sudden changes in supply. figures
\ref{fig:wind_actual_to_activation} and \ref{fig:wind_share_to_activation} show 
values of realized wind production and wind share (wind production as a fraction 
of total production) plotted against the distribution of mFRR activations. These
figures indicate that there is no direct correlation between wind production and
mFRR activations. The existence of such a correlation would have made it easy for
the model to leverage wind production features for predicting activations. The 
hope is, however, that wind production features will prove useful when combined
with other features, as the model captures complex relationships between features.

\begin{figure}[H]
    \centering
    \begin{subfigure}[t]{0.49\textwidth}
        \centering
        \includegraphics[width=\linewidth]{Images/actual_wind_doesnt_impact_prediction.png}
        \caption{Realized wind production plotted against mFRR activation distribution.}
        \label{fig:wind_actual_to_activation}
    \end{subfigure}
    \hfill
    \begin{subfigure}[t]{0.49\textwidth}
        \centering
        \includegraphics[width=\linewidth]{Images/wind_share_impact_activation.png}
        \caption{Forecasted wind production plotted against mFRR activation distribution.}
        \label{fig:wind_share_to_activation}
    \end{subfigure}
    \caption{Load and production data distributions.}
    \label{fig:load_and_production_distributions}
\end{figure}




\subsubsection{Interaction features}
Interaction features are created by combining two or more existing features
to capture complex relationships that may influence mFRR activations.
