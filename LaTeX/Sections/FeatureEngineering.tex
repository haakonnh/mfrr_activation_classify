
\subsection{Feature Engineering}
Features are attributes in a dataset that describe each data point. A dataset
for predicting a person's income could, for instance, have features like
gender, job type, and age. Then, each data point represents a person and
relevant information about the person in terms of the target variable
-- income. In this project, each data point represents a specific time stamp
in the mFRR activation dataset, and the features describe the system state at
that time. This includes information such as electricity demand, generation
capacity, and market prices, all of which can influence mFRR activations.

Already available features can be transformed to create new higher-level
features that may better capture the underlying patterns in the data. For
example, if one has features for year of death and year of birth, a new feature
for age at death can be created by subtracting the year of birth from the
year of death. This is known as \textit{feature construction} 
\autocite{liuFeatureSelectionKnowledge2012}. Features can be constructed in 
various ways, such as through mathematical operations or aggregations. This
project leverages this concept to create new features that may enhance the model's
predictive capabilities.

Feature selection is crucial. There are many features that may seem useful
and relevant in isolation, but sometimes they mislead the models, or they
work poorly in combination with other seemingly good features. Theoretical
analysis of the usefulness of certain features can be helpful, but only
trial-and-error together with feature-importance analysis will uncover the
features' actual impact on performance. Feature selection will be subject to
restrictions outlined in Section \ref{sec:restrictions} to ensure that only
real-time available features are used. 

\subsubsection{Time Restrictions and Data Availability}
Short-term activation prediction is constrained by the limited availability of
recent system data at prediction time. mFRR EAM bidding for a specific MTU closes 45
minutes prior, so at time $t$ we can only act on predictions for the interval
$t+4$ and later. This restriction causes a great amount of uncertainty. Even 
the best approximation of the current system state is several 15-minute intervals old
at the target delievery time. 

Many data sources have reporting delays, meaning that the most recent data points
are not yet published at prediction time. For example, mFRR EAM activation data is
available with a delay of approximately one hour. This means that at time $t$, the
most recent activation data available is from time $t-4$. However, since mFRR EAM 
activations
are known to be highly autocorrelated 
(\autocite{backePredictionsPricesVolumes2023}\textbf{source or maybe not here?}), so 
even delayed activation data can provide valuable information about current trends.


\textbf{Time delay figure?}

Additional challenges arise from data sources published at lower resolutions
resolutions. For example, aFRR activation data is released hourly, causing new
information to arrive in batches rather than continuously. Depending on the
prediction time within the hour, some recent lagged values may not yet exist,
leading to inconsistencies in which features are available in real time.

Similar delays affect other system indicators such as consumption, production,
and wind forecasts, many of which are published with reporting lags. As a
consequence, the prediction task must rely primarily on features that are
available at or before time $t$, including forecasts and lagged values that are
consistently observable. These constraints shape both the feature engineering
process and the achievable model performance, as the model cannot learn from
information that would not be accessible to market participants in practice.


\subsubsection{Lag features}
A lag feature is a feature that represents the value of a variable at a
previous time step. Lag features are commonly used in time series analysis to
capture \textit{temporal} dependencies and trends in the data 
\autocite{brownleeBasicFeatureEngineering2016}. By including lag
features, the model can leverage historical information to make more informed
predictions about future mFRR activations. This becomes increasingly important
when real-time data is limited, as lag features can provide context about recent
system states and patterns.
d
\textbf{Lag feature figure?}

\paragraph{Activation lag features}
Activation lag features are the most important lag features for this problem,
as they convey important information about recent temporal activation trends. 
They are,
however, restricted by the real-time limitations, so the model may only use
activation lag features from $t-4$ and earlier for predicting activations at
time $t+4$. As a result, lag features for upregulating and downregulating
activations are created for time steps $t-4$, $t-5$, $t-6$, ..., $t-9$. These
features are useful by themselves, but they also serve as a basis for creating
other features that capture activation trends more effectively.

\paragraph{Persistence (activation streak length)}
Lag features indicate what happened in the most recent intervals, but they do
not express whether the system has been in a sustained activation phase. For the
model to understand such trends, it would need to look at several lagged
activation features simultaneously and infer whether there has been a streak of
up- or down-activations. To capture this behaviour succinctly, a set of
persistence features is included. These measure how long the latest sequence of
up-, down-, or no-activations has lasted, based only on the activation data that
are available at bid time.  The idea is straightforward: if down-activations
occurred in several consecutive intervals leading up to and including $t-4$, the
down-persistence value reflects the length of that streak, as visualized in
figure \ref{fig:persistence_illustration}. The same applies for
up-activations and no-activations. These persistence features aim to
capture the tendency for mFRR activations to occur in clusters, as periods of
system stress often lead to repeated activations. By condensing this pattern
into a single value for each direction, the model is given a clearer
representation of ongoing activation dynamics than lagged indicators alone can
provide.

\begin{figure}[H]
    \centering
    \includegraphics[width=0.8\textwidth]{Images/Persistence.png}
    \caption{Illustration of down persistence feature calculation based on lagged
    activation features. Here, the down persistence at time $t$ is 2, as there have 
    been down-activations in the two most recent intervals ($t-4$ and $t-5$),
    before an interval with no activation at $t-6$.}
    \label{fig:persistence_illustration}
\end{figure}

Persistence feature capture consecutive activation trends effectively, but
they do have limitations. Singular lagged activation featurs are still useful 
in the case of intermittent activations that do not form long streaks. Additionally,
persistence features do not convey the magnitude of recent activations, only
their occurrence. How all mFRR activation-related features are used in conjunction 
will be subject to experimentation and analysis during model development.


\subsubsection{Cross-zonal flow features}
Cross-zonal flow features capture information about electricity flows between
different zones or regions in the power grid - in this case, in and out of
the NO1 bidding zone. These flows can indicate the grid's stress level and
influence mFRR activations. For instance, high inflows into NO1 may signal
increased demand or generation shortages, potentially causing upregulating
activations. Conversely, high outflows may indicate surplus generation,
potentially causing downregulating activations. It is unlikely, though, that
cross-zonal flow features alone can predict activations. Combining them with
available transfer capacity provides a picture of how close the grid is to its
operational limits. For instance, if the inflow into NO1 is close to the maximum
available transfer capacity, only a small margin remains for additional inflows,
which could increase the likelihood of upregulating activations. Such situations
often occur in zones that are short, i.e. zones where consumption exceeds
production. In such cases, the grid operator may need to activate expensive
mFRR reserves to maintain grid stability when no more cheap imports are possible.
It is important that the models developed in this project are able to capture
these kind of relationships as they are among the most valuable for a potential
user of the models.

\paragraph{Capacity-normalized cross-zonal flow.} Raw cross-zonal flow
magnitudes are not comparable across interconnections or over time because each
line has different capacity and the available transfer capacity (ATC) varies.
The same absolute flow can be insignificant on a high-capacity interconnection
but critical on a constrained one. To obtain a dimensionless,
capacity-normalized measure of proximity to operational limits, and to make
features comparable across borders and time, flows are expressed as a ratio to
the relevant directional ATC.

Let $F_i(t)$ be the flow on interconnection $i$ at time $t$, taken as positive
when power flows into NO1 and negative when it flows out. Let $ATC_i(t)$ be the
available transfer capacity for that interconnection at time $t$. The
capacity-normalized flow is then
\[
    F_{\text{ratio}}^{i}(t) = \frac{F_i(t)}{ATC_i(t)}.
\]
Values of $F_{\text{ratio}}^{i}(t)$ close to $1$ mean that the inflow is close
to the capacity, values close to $-1$ mean that the outflow is close to the
capacity, and values near $0$ mean that the net flow is small compared to the
available capacity.


\subsubsection{Temporal features}
Temporal features capture time-related patterns in the data. These features help the model
understand how mFRR activations vary with time, such as daily or weekly cycles. 
Basic temporal features include hour of the day, day of the week, and month of the year.
These features allow the model to learn patterns related to specific times. mFRR activations
could, for instance, be caused by completely different factors during peak hours on weekdays
compared to off-peak hours on weekends. Temporal features like these are most often
represented using cyclical encoding to reflect their periodic nature. For example,
1 AM and 11 PM are close in time, even though their numerical representations (1 and 23)
are far apart. Cyclical encoding uses sine and cosine transformations to capture this
periodicity \autocite{pelletierCyclicalEncodingAlternative2024}. 
Hourly features are, for instance, encoded as:
\[
\text{Hour}_{\text{sin}} = \sin\left(2\pi \cdot \frac{\text{Hour}}{24}\right),\qquad
\text{Hour}_{\text{cos}} = \cos\left(2\pi \cdot \frac{\text{Hour}}{24}\right).
\]
Monthly features can be encoded similarly, using 12 as the divisor instead of 24.

\subsubsection{Price features}
Price features capture information about the various electricity market prices. 
The mFRR activaiton market is closely linked to other electricity makets, such as the 
day-ahead market, the intraday market, and the aFRR market. Most prices may not have 
direct impacts on activations, but by crafting features that capture important 
relationships between prices, the model may be able to infer system stress levels
that could lead to mFRR activations. Large discrepencies between day-ahead prices
and intraday prices may, for instance, indicate unexpected changes in supply or demand,
which should correlate with mFRR activations. Similarly, the difference between aFRR prices
and mFRR prices may provide insights into the relative costs of balancing services, 
which could influence activation decisions.



\subsubsection{Production features}
Production features capture information about electricity generation, particularly
from renewable sources like wind power. Wind power production features were
considered promising candidates for predicting mFRR activations, as wind power is
intermittent and can cause sudden changes in supply. figures
\ref{fig:wind_actual_to_activation} and \ref{fig:wind_share_to_activation} show 
values of realized wind production and wind share (wind production as a fraction 
of total production) plotted against the distribution of mFRR activations. These
figures indicate that there is no direct correlation between wind production and
mFRR activations. The existence of such a correlation would have made it easy for
the model to leverage wind production features for predicting activations. The 
hope is, however, that wind production features will prove useful when combined
with other features, as the model captures complex relationships between features.

\begin{figure}[H]
    \centering
    \begin{subfigure}[t]{0.49\textwidth}
        \centering
        \includegraphics[width=\linewidth]{Images/actual_wind_doesnt_impact_prediction.png}
        \caption{Realized wind production plotted against mFRR activation distribution.}
        \label{fig:wind_actual_to_activation}
    \end{subfigure}
    \hfill
    \begin{subfigure}[t]{0.49\textwidth}
        \centering
        \includegraphics[width=\linewidth]{Images/wind_share_impact_activation.png}
        \caption{Forecasted wind production plotted against mFRR activation distribution.}
        \label{fig:wind_share_to_activation}
    \end{subfigure}
    \caption{Production data distributions.}
    \label{fig:load_and_production_distributions}
\end{figure}

\subsubsection{Load features}
Load features capture information about electricity consumption patterns.
Absolute consumption magnitude for NO1 is included as a feature, but there are
many ways to encode consumption in normalized or relative terms. For instance,
consumption can be expressed as a ratio to forecasted consumption, to capture
forecast errors. Consumption can also be expressed as a ratio to relevant
historical consumption values. For the final feature set, a ratio between
current consumption and the average consumption at the same hour throughout the
dataset is used as a feature to capture deviations from typical patterns. 


\subsubsection{Interaction features}
Interaction features are created by combining two or more existing features
to capture complex relationships that may influence mFRR activations.

\subsubsection{Feature Correlation}
\textbf{I put this here only to highlight that it might be appropriate to discuss
the possibility that many of the included features correlate with eachother:
eg. changes in cross-zonal flows may correlate with wind forecast errors etc., and
of course the many activation features.}

\textbf{Balancing vs reserve(activaiton vs capacity)}

\textbf{Activation volumes mostly zero, so predicting volumes maybe not
good to predict directly. Two stage.
}

\textbf{Demand side flexibility, dont care about price, marginal costs
zero}

First step towards regression.